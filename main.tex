% Modelo de monografia criado para a Faculdade de Computação
% da Universidade Federal do Pará a partir da classe abntex2.
% Este documento só deverá ser alterado para incluir ou excluir
% elementos pré e pós textuais. Use o comentário do latex (%) caso
% deseje excluir algum elemento.

%% abtex2-modelo-trabalho-academico.tex, v-1.9.6 laurocesar
%% Copyright 2012-2016 by abnTeX2 group at http://www.abntex.net.br/ 
%%
%% This work may be distributed and/or modified under the
%% conditions of the LaTeX Project Public License, either version 1.3
%% of this license or (at your option) any later version.
%% The latest version of this license is in
%%   http://www.latex-project.org/lppl.txt
%% and version 1.3 or later is part of all distributions of LaTeX
%% version 2005/12/01 or later.
%%
%% This work has the LPPL maintenance status `maintained'.
%% 
%% The Current Maintainer of this work is the abnTeX2 team, led
%% by Lauro César Araujo. Further information are available on 
%% http://www.abntex.net.br/
%%
%% This work consists of the files abntex2-modelo-trabalho-academico.tex,
%% abntex2-modelo-include-comandos and abntex2-modelo-references.bib
%%

% ------------------------------------------------------------------------
% ------------------------------------------------------------------------
% abnTeX2: Modelo de Trabalho Academico (tese de doutorado, dissertacao de
% mestrado e trabalhos monograficos em geral) em conformidade com 
% ABNT NBR 14724:2011: Informacao e documentacao - Trabalhos academicos -
% Apresentacao
% ------------------------------------------------------------------------
% ------------------------------------------------------------------------

\documentclass[
	% -- opções da classe memoir --
	12pt,				% tamanho da fonte
	openright,			% capítulos começam em pág ímpar (insere página vazia caso preciso)
	oneside,			% para impressão em frente e verso. Oposto a oneside
	a4paper,			% tamanho do papel.
	% -- opções da classe abntex2 --
	chapter=TITLE,		% títulos de capítulos convertidos em letras maiúsculas
	%section=TITLE,		% títulos de seções convertidos em letras maiúsculas
	%subsection=TITLE,	% títulos de subseções convertidos em letras maiúsculas
	%subsubsection=TITLE,% títulos de subsubseções convertidos em letras maiúsculas
	% -- opções do pacote babel --
	english,			% idioma adicional para hifenização
	french,				% idioma adicional para hifenização
	spanish,			% idioma adicional para hifenização
	brazil				% o último idioma é o principal do documento
	]{abntex2}

% ---
% Pacotes básicos 
% ---
\usepackage{lmodern}			% Usa a fonte Latin Modern
\usepackage{mathptmx}			% Usa a fonte Times New Roman
\usepackage[T1]{fontenc}		% Selecao de codigos de fonte.
\usepackage[utf8]{inputenc}		% Codificacao do documento (conversão automática dos acentos)
\usepackage{lastpage}			% Usado pela Ficha catalográfica
\usepackage{indentfirst}		% Indenta o primeiro parágrafo de cada seção.
\usepackage{color}				% Controle das cores
\usepackage{graphicx}			% Inclusão de gráficos
\usepackage{subcaption}				% Inclusão de gráficos lado a lado
\usepackage{microtype} 			% para melhorias de justificação
\usepackage{tabularx,ragged2e}	% Para inserir tabelas
\usepackage{multirow}			% Para mesclar células
\usepackage[dvipsnames,table,xcdraw]{xcolor}		% Permite adicionar cores nas linhas de tabelas
\usepackage{fancyvrb}			% Permite adicionar arquivos de texto
\usepackage[portuguese, ruled, linesnumbered]{algorithm2e} % Uso de algoritmos
\usepackage{amsfonts}			% Permite usar notação de conjuntos
\usepackage{amsmath}			% Permite citar equações
\usepackage{amsthm}				% Permite criar teoremas e experimentos
\usepackage[font={bf, small}, labelsep=endash, labelfont=bf]{caption}	% Faz legenda de figuras ficarem em negrito
\usepackage{cancel}				% Permite fazer expressão tendendo a zero
\usepackage{epstopdf}			% Converte eps para pdf
\usepackage[final]{pdfpages}

\newcolumntype{L}{>{\RaggedRight\arraybackslash}X}
% ---
		
% ---
% Pacotes adicionais, usados apenas no âmbito do Modelo Canônico do abnteX2
% ---
\usepackage{lipsum}				% para geração de dummy text
% ---

% ---
% Pacotes de citações
% ---
%\usepackage[brazilian,hyperpageref]{backref}	 % Paginas com as citações na bibl
\usepackage[alf, abnt-emphasize=bf]{abntex2cite}	% Citações padrão ABNT

% ---
% Customizações para o layout da UFPA
% ---
\usepackage{modelo-ufpa/ufpa}

% Muda o título de lista de ilustrações para lista de figuras
\addto\captionsbrazil{%
  \renewcommand{\listfigurename}%
    {Lista de Ilustrações}%
	\renewcommand{\listtablename}%
    {Lista de Tabelas}%
}

% Permite utilizar figuras sem precisar colocar o caminho absoluto
\graphicspath{{imagens/}}

% Define o ambiente de experimentos
\theoremstyle{definition}
\newtheorem{experimento}{Experimento}[section]
\newcommand{\experimentoautorefname}{Experimento}

% --- 
% CONFIGURAÇÕES DE PACOTES
% --- 

% ---
% Configurações do pacote backref
% Usado sem a opção hyperpageref de backref
%\renewcommand{\backrefpagesname}{Citado na(s) página(s):~}
% Texto padrão antes do número das páginas
%\renewcommand{\backref}{}
% Define os textos da citação
%\renewcommand*{\backrefalt}[4]{
%	\ifcase #1 %
%		Nenhuma citação no texto.%
%	\or
%		Citado na página #2.%
%	\else
%		Citado #1 vezes nas páginas #2.%
%	\fi}%
% ---

% ---
% Informações de dados para CAPA, FOLHA DE ROSTO e FICHA CATALOGRÁFICA
% ---
\universidade{UNIVERSIDADE FEDERAL DO PARÁ}
\instituto{INSTITUTO DE CIÊNCIAS BIOLÓGICAS}
\faculdade{FACULDADE DE BIOTECNOLOGIA}
\curso{CURSO DE BACHARELADO EM BIOTECNOLOGIA}
\titulo{Predição funcional de genes de resistência em Corynebacterium pseudotuberculosis biovar equi após indução da resistência a rifampicina e enrofloxacina}
\autor{DAVI JOSUÉ MARCON}
\local{Belém}
\data{2022}
\orientador{Prof. Dr. Rafael Azevedo Baraúna}
\tipotrabalho{Monografia}
% O preambulo deve conter o tipo do trabalho, o objetivo, 
% o nome da instituição e a área de concentração 
\preambulo{Trabalho de Conclusão de Curso apresentado para obtenção do grau de Bacharel em Biotecnologia.}
\sobrenome{Marcon}
\nome{Davi}
\palavraschave{%
1. Bioinformática.%TODO
2. Curadoria de genomas. %TODO
3. Fechamento de gaps. %TODO
}
\datadadefesa{Data da Defesa: }%TODO
\conceito{Conceito: }%TODO
\faculdadedoorientador{Faculdade de Biotecnologia - UFPA}
\primeiromembrodabanca{Prof. Dr. Nome Sobrenome} %TODO
\faculdadedoprimeiromembrodabanca{Faculdade de Computação - UFPA}%TODO
\segundomembrodabanca{Prof. Dra. Nome Sobrenome}%TODO
\faculdadedosegundomembrodabanca{Faculdade de Biotecnologia - UFPA}%TODO
% ---


% ---
% Configurações de aparência do PDF final

% alterando o aspecto da cor azul
\definecolor{blue}{RGB}{41,5,195}

% informações do PDF
\makeatletter
\hypersetup{
     	%pagebackref=true,
		pdftitle={\imprimirtitulo}, 
		pdfauthor={\imprimirautor},
    	pdfsubject={\imprimirpreambulo},
	    pdfcreator={LaTeX with abnTeX2},
		pdfkeywords={\imprimirpalavraschave}, 
		colorlinks=true,       		% false: boxed links; true: colored links
    	linkcolor=black,          	% color of internal links
    	citecolor=black,        		% color of links to bibliography
    	filecolor=magenta,      		% color of file links
		urlcolor=blue,
		bookmarksdepth=4,
        breaklinks=true
}
\makeatother
% --- 

% --- 
% Espaçamentos entre linhas e parágrafos 
% --- 

% O tamanho do parágrafo é dado por:
\setlength{\parindent}{1.3cm}

% Controle do espaçamento entre um parágrafo e outro:
\setlength{\parskip}{0.2cm}  % tente também \onelineskip

% ---
% compila o indice
% ---
\makeindex
% ---

% ----
% Início do documento
% ----
\begin{document}

% Seleciona o idioma do documento (conforme pacotes do babel)
%\selectlanguage{english}
\selectlanguage{brazil}

% Retira espaço extra obsoleto entre as frases.
\frenchspacing 

% ----------------------------------------------------------
% ELEMENTOS PRÉ-TEXTUAIS
% ----------------------------------------------------------
% \pretextual

% ---
% Capa
% ---
\imprimircapa
% ---

% ---
% Folha de rosto
% ---
\imprimirfolhaderosto
% ---

% ---
% Inserir a ficha bibliografica
% ---

% Isto é um exemplo de Ficha Catalográfica, ou ``Dados internacionais de
% catalogação-na-publicação''. Você pode utilizar este modelo como referência. 
% Porém, provavelmente a biblioteca da sua universidade lhe fornecerá um PDF
% com a ficha catalográfica definitiva após a defesa do trabalho. Quando estiver
% com o documento, salve-o como PDF no diretório do seu projeto e substitua todo
% o conteúdo de implementação deste arquivo pelo comando abaixo:
%
% \begin{fichacatalografica}
%     \includepdf{fig_ficha_catalografica.pdf}
% \end{fichacatalografica}

\newpage

\begin{fichacatalografica}
	\imprimirfichacatalografica
\end{fichacatalografica}
% ---

% ---
% Inserir errata
% ---
\begin{errata}

Elemento opcional da \citeonline[4.2.1.2]{NBR14724:2011}. Exemplo:

\vspace{\onelineskip}


\begin{table}[htb]
\center
\footnotesize
\begin{tabular}{|p{1.4cm}|p{1cm}|p{3cm}|p{3cm}|}
  \hline
   \textbf{Folha} & \textbf{Linha}  & \textbf{Onde se lê}  & \textbf{Leia-se}  \\
    \hline
    1 & 10 & auto-conclavo & autoconclavo\\
   \hline
\end{tabular}
\end{table}


\end{errata}
% ---

% ---
% Inserir folha de aprovação
% ---

% Isto é um exemplo de Folha de aprovação, elemento obrigatório da NBR
% 14724/2011 (seção 4.2.1.3). Você pode utilizar este modelo até a aprovação
% do trabalho. Após isso, substitua todo o conteúdo deste arquivo por uma
% imagem da página assinada pela banca com o comando abaixo:
%
% \includepdf{folhadeaprovacao_final.pdf}
%
\begin{folhadeaprovacao}
\imprimirfolhadeaprovacao
\end{folhadeaprovacao}
% ---

% ---
% Dedicatória
% ---
\begin{dedicatoria}

\vspace*{\fill}
   \centering
   \noindent
   \textit{Este trabalho é dedicado a todos aqueles que,\\
   de alguma forma abdicaram de algo e\/ou a si mesmos\\
   pela Ciência.} \vspace*{\fill}


\end{dedicatoria}
% ---

% ---
% Agradecimentos
% ---

\begin{agradecimentos}
Pela orientação e angario de financiamento: Rafael Baraúna, Paula Schneider, Arthur Silva. 

A equipe do Centro de Genômica e Biologia de Sistemas: Beatriz Lobato, Carolina Miranda, Soraya Andrade,
Silvanira Barbosa. 

Aos desenvolvedores, usuários e contribuintes dos projetos \abnTeX, \LaTeX, nextflow, powershell, 
prokka, antismash, blast, fastq, multiqc, armfinder, ragout, busco, kraken, artemis e demais projetos
associados.

Alexandra Elbakyan e ao(s) fundador(es) da Genesis Library.

Aos meus pais e meus irmãos: Fábio, Silvane, Israel, Giulia e Laura Marcon.

Aos meus amigos e colegas: Cayo Uchôa, Tiago Moura, Gustavo Marques, Joaquim Neto, Thiago Cordeiro, Naiana Ribeiro, 
Valéria Silva, Giovane Pinheiro, Gabriela Campestrini, Wictoria Dias, Aline Castro, Mayanne Farias, Amanda Oliveira, 
Letícia Lago, José Lucas, Iago Blanco, Isabel Montoril, Gabrielly Andrade, Mayza Miranda,
Beatriz Moura, Beatriz Campos. 

A minha namorada: Clara Feitosa.

Colaboradores externos a minha formação: Emilyn Conceição, Abhinav Sharma, Marília Lima, Karla Lima, Alex Souza, Robert Petit III.

Aos professores: Luciana Xavier, Agenor Valadares, Rommel Jucá, Moysés Miranda, Adriana Folador,
Bruno Duarte, Ricardo de Deus e Alejandro Prado.

Pela estrutura: Universidade Federal do Pará, Centro de Gênomica e Biologia de Sistemas, ENGBIO.

As fontes de financiamento: CAPES, CNPQ, FAPESPA, UFPA.


Nenhum trabalho é feito sozinho.


Obrigado.

\end{agradecimentos}


% ---

% ---
% Epígrafe
% ---
\begin{epigrafe}
    
\vspace*{\fill}
	\begin{flushright}
		\textit{``A verdadeira viagem de descobrimento 
        \\não consiste em procurar novas paisagens,
        \\mas em ter novos olhos.''\\
		(Marcel Proust)}
	\end{flushright}


\end{epigrafe}
% ---

% ---
% RESUMOS
% ---

\setlength{\absparsep}{18pt} % ajusta o espaçamento dos parágrafos do resumo
\begin{resumo}

Tendo em mente a atual crise no mercado farmacológico devido ao surgimento de microrganismos
multiresistestes, faz-se necesária a readequação das metodologias de desenvolvimento de fármacos.
A mineração genômica permite predizer a capacidade de microrganismos produzirem
metabólitos sem a necessidade de testes \textit{in vitro}, encurtando os passos
até a descoberta de novos fármacos. A partir do sequenciamento de duas cepas
bacterianas $($\textit{Rhodococcus sp.} e \textit{Brevibacillus brevis} $)$, foi 
possível montar seu genomas utilizando a ferramenta SPADES, predizer os genes nos
genomas utilizando a ferramenta PROKKA e predizer a produção de Metabólitos 
secundários usando o ANTI-SMASH.  Como principais resultados obtivemos que a cepa
de \textit{Rhodococcus sp.}, observamos a presença de 16 clusters ainda sem a função definida.
A amostra \textit{Brevibacillus brevis} apresentou 15 clusters sendo 3 $($macrobrevina,tirocidina e gramicidina$)$ com função predita
para atividade antimicrobiana. A técnica de mineração genômica, permitiu prospectar informações
a respeito da produção de metabóltios, com isso foi possível avaliar o potencial biotecnológico
desses organismos com técnicas independentes de cultivo e independente dos 
padrões de expressão do organismo

\vspace{\onelineskip}
\noindent 
\textbf{Palavras-chave}: Bactérias; Potencial Biotecnológico; Genômica; Predição computacional.

\end{resumo}

% resumo em inglês
\begin{resumo}[Abstract]
 \begin{otherlanguage*}{english}

   Having in mind the current crisis in the pharmacological market due to the emergence of
   multidrug resistat microorganisms, it is necessary to readjust the methodologies of drug development.
   Genomic mining makes possible to predict the ability of microorganisms to produce
   metabolites without the need for \textit{in vitro} tests, shortening the steps
   until the discovery of new drugs. From the sequencing of two
   bacterial strains $($\textit{Rhodococcus sp.} and \textit{Brevibacillus brevis} $)$, we
   assembled their genomes using SPADES, predicted the genes in the
   genomes using the PROKKA and predicted the production of Metabolites
   secondary using ANTI-SMASH. As main results we obtained that the strain
   of \textit{Rhodococcus sp.}, we observed the presence of 16 clusters without a defined function.
   The sample \textit{Brevibacillus brevis} showed 15 clusters, 3 $($macrobrevin, thyrocidin and gramicidin$)$ with predicted function
   for antimicrobial activity. The genomic mining technique allowed us to prospect information
   regarding the production of metabolites, it was possible to evaluate the biotechnological potential
   of these organisms with independent cultivation techniques and independently of the
   expression patterns of the organism

   \vspace{\onelineskip}
   \noindent 
   \textbf{Keywords}: Bacterias; Biotechnological Potential; Genomic; Computational Prediction.
 \end{otherlanguage*}
\end{resumo}

% resumo em francês 
%\begin{resumo}[Résumé]
% \begin{otherlanguage*}{french}
%    Il s'agit d'un résumé en français.
% 
%   \textbf{Mots-clés}: latex. abntex. publication de textes.
% \end{otherlanguage*}
%\end{resumo}

% resumo em espanhol
%\begin{resumo}[Resumen]
% \begin{otherlanguage*}{spanish}
%   Este es el resumen en español.
%  
%   \textbf{Palabras clave}: latex. abntex. publicación de textos.
% \end{otherlanguage*}
%\end{resumo}


% ---
% Listas de siglas, tabelas e simbolos
% ---

% ---
% inserir lista de ilustrações
% ---
\pdfbookmark[0]{\listfigurename}{lof}
\listoffigures*
\cleardoublepage
% ---

% ---
% inserir lista de tabelas
% ---
\pdfbookmark[0]{\listtablename}{lot}
\listoftables*
\cleardoublepage
% ---

% ---
% inserir lista de abreviaturas e siglas
% ---
\begin{siglas}
	\item[BGC] \textit{Biosyntetic Genes Cluster } - Cluster de Genes Biossintéticos
  \item[AMP] \textit{Antimicrobial Peptides} - Peptídeos Antimicrobianos
  \item[MIB] Microrganismos de Interesse Biotecnológico
  \item[NCBI] \textit{National Center of Biotechnology Information}
  \item[DNA] \textit{Deoxyribonucleic acid} - Ácido Desoxirribonucleico
  \item[RNA] \textit{Ribonucleic acid} - Ácido ribonucleico
  \item[ARG] \textit{Antibiotic Resistance Gene} - Gene de resistência a antibiótico
\end{siglas}
% ---


% ---
% inserir o sumario
% ---
\pdfbookmark[0]{\contentsname}{toc}
\tableofcontents*
\cleardoublepage
% ---



% ----------------------------------------------------------
% ELEMENTOS TEXTUAIS
% ----------------------------------------------------------
\textual





% ----------------------------------------------------------
% Introdução
% ----------------------------------------------------------
\chapter{Introdução}
\label{cap:introducao}

\section{Contexto}

- Necessidade de novos Compostos

- Uso de Biotecnologia para solução de problemas industriais

- Diversidade amazônica como reservatório de descobertas


\section{Justificativa}
Bactérias ambientais são interessantes alvos para a descoberta de compostos
de relevância biotecnológica, especialmente como solução para os crescentes níveis
de resistência a antimicrobianos encontrados em microorganismos patogênicos.
A caracterização genômica e prospecção de genes de interesse desses microorganismos,
especialmente do ambiente amazônico, são passos importantes
em busca de compostos de potencial farmacológico e industrial.

\chapter{Objetivos}

\section{Objetivo Geral}

Predizer o potencial biotecnológico de bactérias ambientais utilizando 
ferramentas \textit{in silico} 

\section{Objetivos Específicos}
\begin{enumerate}
    \item Caracterizar os organismos sequênciados utilizando seus genomas
    \item Predizer as características metabólicas dos organismos
    \item Categorizar os microorganismos quanto a capacidade de produção de compostos de interesse biotécnológico
\end{enumerate}






% ----------------------------------------------------------
% Referenciais Teóricos
% ----------------------------------------------------------
\chapter{Referenciais Teóricos}
\label{cap:referenciais_teoricos}




\section{\textit{C. pseudotuberculosis }}
\subsection{Linfadenite Caseosa}
\subsection{Biovares}
\subsection{Níveis de susceptibilidade a antibióticos}
\section{Mutações como mecanismo de resistência em gram negativos}
\subsection{Modificação em sítio}
\section{Resistência induzida por doses sub-letais}

% ----------------------------------------------------------
% Resultados e Discussão
% ----------------------------------------------------------
\chapter{Resultados}
\label{cap:resultados}

\begin{table}[!htb]
\centering
\caption{Informações de montagem dos genomas de referência}
\label{tab:informacoes_montagem}
\begin{tabular}{cccc}
\toprule
\textbf{Organismo} & \textbf{Montador} & \textbf{Bases com N} & \textbf{\textit{Scaffolds}} \\ \midrule
\rowcolor[HTML]{F3F3F3} 
\textbf{\textit{Corynebacterium pseudotuberculosis 262}} & & & \\
\rowcolor[HTML]{DBDEDE} 
 & \textit{SPADES} & 2893857 & 4611 \\
\rowcolor[HTML]{F3F3F3} 
\textbf{\textit{Staphylococcus aureus A--S391\_USA300}} & & & \\
\rowcolor[HTML]{DBDEDE} 
 & \textit{ABySS} & 3893185 & 5012 \\
\rowcolor[HTML]{F3F3F3} 
 & \textit{ABySS2} & 3821622 & 125 \\
\rowcolor[HTML]{DBDEDE} 
 & \textit{Allpaths-LG} & 2880676 & 19 \\
\rowcolor[HTML]{F3F3F3} 
 & \textit{Bambus2} & 2862930 & 17 \\
\rowcolor[HTML]{DBDEDE} 
 & MSR-CA & 2872905 & 17 \\
\rowcolor[HTML]{F3F3F3} 
 & SGA & 3128388 & 546 \\
\rowcolor[HTML]{DBDEDE} 
 & \textit{SOAPdenovo} & 2924135 & 175 \\
\rowcolor[HTML]{F3F3F3} 
 & \textit{Velvet} & 2877995 & 173 \\
\rowcolor[HTML]{DBDEDE} 
\textbf{\textit{Rhodobacter sphaeroides 2.4.1}} & & & \\
\rowcolor[HTML]{F3F3F3} 
 & \textit{ABySS} & 5160167 & 2714 \\
\rowcolor[HTML]{DBDEDE} 
 & \textit{ABySS2} & 5331930 & 480 \\
\rowcolor[HTML]{F3F3F3} 
 & \textit{Allpaths-LG} & 4609785 & 38 \\
\rowcolor[HTML]{DBDEDE} 
 & \textit{Bambus2} & 4428612 & 92 \\
\rowcolor[HTML]{F3F3F3} 
 & \textit{CABOG} & 4259679 & 130 \\
\rowcolor[HTML]{DBDEDE} 
 & MSR-CA & 4498559 & 44 \\
\rowcolor[HTML]{F3F3F3} 
 & SGA & 5614693 & 2096 \\
\rowcolor[HTML]{DBDEDE} 
 & \textit{SOAPdenovo} & 4627058 & 312 \\
\rowcolor[HTML]{F3F3F3} 
 & \textit{Velvet} & 4615068 & 382 \\ \bottomrule \\
\end{tabular}
\begin{small}\textbf{Fonte: \citeonline{gapblaster2016}}\end{small}
\end{table}

\begin{quadro}[!htb]
	\centering
	\caption{Comparação das funcionalidades do \textit{GapBlaster}, \textit{FGAP} e \textit{GapFiller}}
	\label{quadro:comparacao_funcionalidades}
	\resizebox{\textwidth}{!}{\begin{tabular}{lccc}
			\toprule
			\multicolumn{1}{c}{\textbf{Funcionalidades}} & \textbf{\textit{GapBlaster}} & \textbf{\textit{FGAP}} & \textbf{\textit{GapFiller}} \\ \midrule
			\rowcolor[HTML]{F3F3F3} 
			Método de alinhamento & \textit{Legacy Blast}, \textit{Blast+} ou \textit{MUMmer} & \textit{Blast+} & \textit{Bowtie} ou BWA \\
			\rowcolor[HTML]{DBDEDE} 
			Ajuste do tamanho da região flanqueadora & Sim & Sim & Sim \\
			\rowcolor[HTML]{F3F3F3} 
			Permite curadoria manual & Sim & Não & Não \\
			\rowcolor[HTML]{DBDEDE} 
			Realiza análise automática & Sim & Sim & Sim \\
			\rowcolor[HTML]{F3F3F3} 
			Usa leituras pareadas para fechar \textit{gaps} & Não & Não & Sim \\
			\rowcolor[HTML]{DBDEDE} 
			Usa \textit{contigs} para fechar \textit{gaps} & Sim & Sim & Não \\
			\rowcolor[HTML]{F3F3F3} 
			Lê arquivos nos formatos FASTQ, SAM e BAM & Não & Não & Sim \\
			\rowcolor[HTML]{DBDEDE} 
			Executa código em paralelo & Não & Sim & Não \\
			\rowcolor[HTML]{F3F3F3} 
			Interface gráfica & Sim & Não & Não \\
			\rowcolor[HTML]{DBDEDE} 
			Melhora o resultado de outros programas & Sim & Não foi testado & Não foi testado \\
			\rowcolor[HTML]{F3F3F3} 
			Fecha \textit{gaps} corretamente & Sim & Sim & Sim \\ \bottomrule \\
	\end{tabular}}
	\begin{small}\textbf{Fonte: \citeonline{gapblaster2016}}\end{small}
\end{quadro}


% ----------------------------------------------------------
% Considerações Finais
% ----------------------------------------------------------
\input{textuais/03_considerações_finais.tex}
% ----------------------------------------------------------
% ELEMENTOS PÓS-TEXTUAIS
% ----------------------------------------------------------
\postextual
% ----------------------------------------------------------

% ----------------------------------------------------------
% Referências bibliográficas
% ----------------------------------------------------------
\bibliography{bibliografia}
% ---


% ----------------------------------------------------------
% Apêndices
% ----------------------------------------------------------

% ---
% Inicia os apêndices
% ---
\begin{apendicesenv}
	
	% Imprime uma página indicando o início dos apêndices
	\partapendices
	
% ----------------------------------------------------------
\chapter{Quisque libero justo}
% ----------------------------------------------------------

\lipsum[50]

% ----------------------------------------------------------
\chapter{Nullam elementum urna}
% ----------------------------------------------------------
\lipsum[55-57]
	
\end{apendicesenv}
% ---

% ----------------------------------------------------------
% Anexos
% ----------------------------------------------------------

% ---
% Inicia os anexos
% ---
\begin{anexosenv}
	
	% Imprime uma página indicando o início dos anexos
	\partanexos
	
	% ---
    \chapter{Morbi ultrices rutrum lorem}
    % ---
    \lipsum[30]

    % ---
    \chapter{Cras non urna sed}
    % ---

    \lipsum[31]

    % ---
    \chapter{Fusce facilisis lacinia dui}
    % ---

    \lipsum[32]
	
\end{anexosenv}

\end{document}
