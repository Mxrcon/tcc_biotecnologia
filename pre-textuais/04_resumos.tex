\setlength{\absparsep}{18pt} % ajusta o espaçamento dos parágrafos do resumo
\begin{resumo}

Tendo em mente a atual crise no mercado farmacológico devido ao surgimento de microrganismos
multiresistestes, faz-se necesária a readequação das metodologias de desenvolvimento de fármacos.
A mineração genômica permite predizer a capacidade de microrganismos produzirem
metabólitos sem a necessidade de testes \textit{in vitro}, encurtando os passos
até a descoberta de novos fármacos. A partir do sequenciamento de duas cepas
bacterianas $($\textit{Rhodococcus sp.} e \textit{Brevibacillus brevis} $)$, foi 
possível montar seu genomas utilizando a ferramenta SPADES, predizer os genes nos
genomas utilizando a ferramenta PROKKA e predizer a produção de Metabólitos 
secundários usando o ANTI-SMASH.  Como principais resultados obtivemos que a cepa
de \textit{Rhodococcus sp.}, observamos a presença de 16 clusters ainda sem a função definida.
A amostra \textit{Brevibacillus brevis} apresentou 15 clusters sendo 3 $($macrobrevina,tirocidina e gramicidina$)$ com função predita
para atividade antimicrobiana. A técnica de mineração genômica, permitiu prospectar informações
a respeito da produção de metabóltios, com isso foi possível avaliar o potencial biotecnológico
desses organismos com técnicas independentes de cultivo e independente dos 
padrões de expressão do organismo

\vspace{\onelineskip}
\noindent 
\textbf{Palavras-chave}: Bactérias; Potencial Biotecnológico; Genômica; Predição computacional.

\end{resumo}

% resumo em inglês
\begin{resumo}[Abstract]
 \begin{otherlanguage*}{english}

   Having in mind the current crisis in the pharmacological market due to the emergence of
   multidrug resistat microorganisms, it is necessary to readjust the methodologies of drug development.
   Genomic mining makes possible to predict the ability of microorganisms to produce
   metabolites without the need for \textit{in vitro} tests, shortening the steps
   until the discovery of new drugs. From the sequencing of two
   bacterial strains $($\textit{Rhodococcus sp.} and \textit{Brevibacillus brevis} $)$, we
   assembled their genomes using SPADES, predicted the genes in the
   genomes using the PROKKA and predicted the production of Metabolites
   secondary using ANTI-SMASH. As main results we obtained that the strain
   of \textit{Rhodococcus sp.}, we observed the presence of 16 clusters without a defined function.
   The sample \textit{Brevibacillus brevis} showed 15 clusters, 3 $($macrobrevin, thyrocidin and gramicidin$)$ with predicted function
   for antimicrobial activity. The genomic mining technique allowed us to prospect information
   regarding the production of metabolites, it was possible to evaluate the biotechnological potential
   of these organisms with independent cultivation techniques and independently of the
   expression patterns of the organism

   \vspace{\onelineskip}
   \noindent 
   \textbf{Keywords}: Bacterias; Biotechnological Potential; Genomic; Computational Prediction.
 \end{otherlanguage*}
\end{resumo}

% resumo em francês 
%\begin{resumo}[Résumé]
% \begin{otherlanguage*}{french}
%    Il s'agit d'un résumé en français.
% 
%   \textbf{Mots-clés}: latex. abntex. publication de textes.
% \end{otherlanguage*}
%\end{resumo}

% resumo em espanhol
%\begin{resumo}[Resumen]
% \begin{otherlanguage*}{spanish}
%   Este es el resumen en español.
%  
%   \textbf{Palabras clave}: latex. abntex. publicación de textos.
% \end{otherlanguage*}
%\end{resumo}
