\setlength{\absparsep}{18pt} % ajusta o espaçamento dos parágrafos do resumo
\begin{resumo}
Microrganismos ambientais são fontes interessantes de genes e metabólitos
com uso biotecnológico, esses genes são comumente encontrados em clusters de genes biossintéticos $($BGC$)$
que são aglomerados gênicos contendo genes que codificam diversas enzimas importantes
para o processo de síntese dos metabólitos. Nem sempre BGC's são expressos e permanecem 
silenciosos em condições laboratoriais, sendo necessário o uso de técnicas de eliciação
da expressão ou o estudo genômico para analisar e descrever a presença de vias biossintéticas.

Este trabalho utilizou dois microorganismos da coleção microbiológica do
Centro de Genômica e biologia de sistemas, e através de sequênciamento
e análise genômica, buscou predizer a capacidade desses microorganismo
para produzir  compostos de interesse biotecnológico.

Como principais resultados obtivemos que a cepa 094 possui a maquinaria
capaz de produzir ao menos três compostos com possível atividade antimicrobiana, 
além de possuir vários genes de resistência associados a produção desses compostos.

\textbf{Palavras-chave}:Bactérias, Potêncial biotecnológico, Gênomica, Predição computacional
\end{resumo}

% resumo em inglês
%\begin{resumo}[Abstract]
% \begin{otherlanguage*}{english}
%   This is the english abstract.
%
%   \vspace{\onelineskip}
% 
%   \noindent 
%   \textbf{Keywords}: latex. abntex. text editoration.
% \end{otherlanguage*}
%\end{resumo}

% resumo em francês 
%\begin{resumo}[Résumé]
% \begin{otherlanguage*}{french}
%    Il s'agit d'un résumé en français.
% 
%   \textbf{Mots-clés}: latex. abntex. publication de textes.
% \end{otherlanguage*}
%\end{resumo}

% resumo em espanhol
%\begin{resumo}[Resumen]
% \begin{otherlanguage*}{spanish}
%   Este es el resumen en español.
%  
%   \textbf{Palabras clave}: latex. abntex. publicación de textos.
% \end{otherlanguage*}
%\end{resumo}
