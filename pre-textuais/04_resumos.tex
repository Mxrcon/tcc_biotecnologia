\setlength{\absparsep}{18pt} % ajusta o espaçamento dos parágrafos do resumo
\begin{resumo}

Tendo em mente a atual crise no mercado farmacológico devido ao surgimento de microrganismos
multiresistestes, faz-se necesária a readequação das metodologias de desenvolvimento de fármacos.
A mineração genômica permite predizer a capacidade de microrganismos produzirem
metabólitos sem a necessidade de testes \textit{in vitro}, encurtando os passos
até a descoberta de novos fármacos. A partir do sequenciamento de duas cepas
bacterianas $($\textit{Rhodococcus sp.} e \textit{Brevibacillus brevis} $)$, foi 
possível montar seu genoma utilizando a ferramenta SPADES, predizer os genes nos
genomas utilizando a ferramenta PROKKA e predizer a produção de Metabólitos 
secundários usando o ANTI-SMASH.  Como principais resultados obtivemos que a cepa
de \textit{Rhodococcus sp.}, observamos a presença de 16 clusters ainda sem a função definida.
A amostra \textit{Brevibacillus brevis} apresentou 15 clusters sendo 3 $($macrobrevina,tirocidina e gramicidina$)$ com função predita
para atividade antimicrobiana. A técnica de mineração genômica, permitiu prospectar informações
a respeito da produção de metabóltios, com isso foi possível avaliar o potencial biotecnológico
desses organismos com técnicas independentes de cultivo.

\vspace{\onelineskip}
\noindent 
\textbf{Palavras-chave}: bactérias, potencial biotecnológico, genômica, predição computacional

\end{resumo}

% resumo em inglês
\begin{resumo}[Abstract]
 \begin{otherlanguage*}{english}
   This is the english abstract.
   \vspace{\onelineskip}
   \noindent 
   \textbf{Keywords}: bacterias, biotechnological potential, genomic, computational prediction.
 \end{otherlanguage*}
\end{resumo}

% resumo em francês 
%\begin{resumo}[Résumé]
% \begin{otherlanguage*}{french}
%    Il s'agit d'un résumé en français.
% 
%   \textbf{Mots-clés}: latex. abntex. publication de textes.
% \end{otherlanguage*}
%\end{resumo}

% resumo em espanhol
%\begin{resumo}[Resumen]
% \begin{otherlanguage*}{spanish}
%   Este es el resumen en español.
%  
%   \textbf{Palabras clave}: latex. abntex. publicación de textos.
% \end{otherlanguage*}
%\end{resumo}
