\chapter{Conclusão}
\label{conclusao}


A montagem dos genomas foi considerada satisfatória para a sugestão de um \textit{draft}, tendo
estatísticas aceitáveis e conteúdo GC próximo ao da literatura para o depósito com numero suficiente de CDSs preditas. Porém para a descrição
de um genoma completo ainda é necessário o uso de tecnologias de fechamento de \textit{gaps} e aprimoramento
das montagens, uma vez que ambas as montagens contem vários espaçamentos de "NNN" em seus genomas.

Como resultados desse trabalho obtivemos a montagem de 2 novos genomas a fim de depositar no GenBank. Além de ter descrito o potêncial biotecnológico da amostra 094
através de ferramentas \textit{in silico}, abrindo espaço para metodologias confirmatórias para
estudo detalhado dos metabólitos desse microrganismo.

Os \textit{BCGS} encontrados na cepa de \textit{Brevibacillus brevis} (94) correspondem a cluster capazes de produzir macrobrevina,
petrobactina,tirocidina e gramicidina. A capacidade de predizer esses cluster, permite o
uso de eliciadores específicos para esse tipo de cluster conforme a metodologia descrita por \citeonline{felicio2021}.

A técnica de mineração genômica, permitiu prospectar informações a respeito
da produção de metabóltios de maneira específica e independente dos padrões de expressão 
do organismo, o que é de grande interesse uma vez que esse tipo de \textit{cluster} gênico costuma permanecer
silenciado em condições de abundância de nutrientes.94

Concluímos que a cepa de \textit{Brevibacillus brevis } possui potencial biotecnológico, apesar de se predita 
como capaz produzir compostos já conhecidos, seus clusters não são idênticos aos presentes nos bancos 
de dados e podem conter modificações estruturais interessantes para o desenvolvimento
de fármacos baseados nessas modificações, ou o aprimoramento de fármacos já existentes.

A cepa de \textit{Rhodococcus sp.} (16), foi predita por possuir diversos \textit{operons}, mas interessantemente
não foi possível encontrar referências para sua função. Esses dados, apesar de não possuírem
impacto biotecnológico direto, são importantes pois através deles, será possível comparar a produção
de metabólitos com os genes e através de estudos de expressão heteróloga descrever a função
desses clusters e complementar os bancos de dados existentes.

Esse estudo buscou estudar o potencial biotecnológico de dois microrganismos do parque estadual, tendo
encontrado resultados importantes em ambas as amostras. Devido a grande quantidade de microrganismos
isolados e depositados no banco de bactérias do Centro de Genômica, podemos imaginar a riqueza 
disponível no genoma desses organismos. O estudo aprofundado do genoma de outros microrganismos pode
permitir o estudo de diversos outros \textit{clusters} biossintéticos, e ajudar a compreender
a matéria escura gênica $($silenciada$)$ desses seres. 

O uso das metodologias \textit{in silico} permitiu a análise detalhada desses organismos, e através desses genomas
outras perguntas ainda  podem ser respondidas como a presença de ilhas de patogenicidade e genes de virulência
além da relação entre esses microrganismos e outros da mesma espécie já descritos anteriormente.

A confirmação da produção de substâncias antimicrobianas será realizada pela equipe do centro de genômica
e Biologia de Sistemas para posterior purificação e descrição de sua estrutura.