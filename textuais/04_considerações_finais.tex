\chapter{Conclusão}
\label{conclusao}

Como resultados desse trabalho obtivemos a montagem de 2 novos genomas a fim de depositar no GenBank. Além de ter descrito o potêncial biotecnológico da amostra 094
através de ferramentas \textit{in silico}, abrindo espaço para metodologias confirmatórias para
estudo detalhado dos metabólitos desse microrganismo. A cepa da amostra 016 apesar de não tem apresentado
clusters identificados ainda deve ser estudada detalhadamente, haja vista que seus metabólitos não puderam
ser identificados mesmo ao utilizar os maiores bancos de dados de \textit{clusters} produtores de produtos naturais disponíveis.

O uso das metodologias \textit{in silico} permite a análise detalhada desses organismos, e através desses genomas
outras perguntas ainda  podem ser respondidas como a presença de ilhas de patogenicidade e genes de virulência
além da relação entre esses microrganismos e outros da mesma espécie já descritos anteriormente.

A confirmação da produção de substâncias antimicrobianas será realizada pela equipe do centro de genômica
e Biologia de Sistemas para posterior purificação e descrição de sua estrutura.