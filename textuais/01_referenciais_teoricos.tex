\chapter{Referenciais Teóricos}
\label{cap:referenciais_teoricos}

\section{Microoganismos de Interesse Biotecnológico (MIB)}
\subsection{Metabolitos secundários e a descoberta de fármacos}

\section{Actinomicetos}
Actinomicetos são um filo de microorganismos gram-positivos de alto conteúdo
guanina e citosina que contém as classes: Acidimicrobiia, Actinobacteria, 
Coriobacteriia, Nitriliruptoria, Rubrobacteria, e Thermoleophilia\cite{yadav2018}.
Dentre suas principais caracteristicas podemos ressaltar a presença de micélios
e a produção de hifas filamentosas \cite{chater2016}.
Essas bactérias foram uma grande fonte para o desenvolvimento de diversos
compostos como:%TODO
e segundo \citeonline{genilloud2017} continuam sendo uma fonte relevante
para o isolamento de caracterização de compostos de interesse biotecnológicos, com o
emprego de metodologias modernas de investigação como a análise gênomica preditiva
podem continuar a fornecer substâncias de relevantes para mercado.

\subsection{Streptomyces}%Citar a daniela por favor
\subsection{Rhodococcus}
\subsection{Kitastospora}
\section{Bacillus}

\section{Estudo genômico de MIB's}
