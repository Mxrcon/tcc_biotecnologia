\chapter{Referenciais Teóricos}
\label{cap:referenciais_teoricos}

\section{Metabolitos secundários e a descoberta de fármacos}
O metabolismo secundário bacteriano é capaz de produzir substâncias de diversas
classes como: alcalóides, carboidratos, esteroídes, lipídeos, peptídeos, policetídeos e
terpenóides. Os quais possuem uma diversa gama de funções, seja como metodologia de "guerra
química" com outros microorganismos, mediadores de atividade mutualística entre espécies,
simbiose química \cite{obrien2011}. Apesar de não serem considerados essenciais para a vida 
desses organismos \cite{demain2009} são de grande importância para sua dispersão e adaptação
em ambientes hostis e excassos de nutrientes. 

\section{Actinomicetos}
Actinomicetos são um filo de microorganismos gram-positivos de alto conteúdo
guanina e citosina que contém as classes: Acidimicrobiia, Actinobacteria, 
Coriobacteriia, Nitriliruptoria, Rubrobacteria, e Thermoleophilia\cite{yadav2018}.
Dentre suas principais caracteristicas podemos ressaltar a presença de micélios
e a produção de hifas filamentosas \cite{chater2016}. Sua dispersão ambiental é enorme
e já foram isolados de ambientes diversos como: lagos salinos, mar profundo e solo \cite{flores2021,felicio2021,sapkota2020}.
Além da simbose com animais, fungos, insetos, línquens e plantas \cite{hei2021,van2017}.
A capacidade de se adaptar a diversos ambientes está intimamente relacionada com a capacidade
de produzir substâncias bioativas com funções igualmente diversas  \cite{van2020}

Essas bactérias foram uma fonte importante para o desenvolvimento de compostos de funções
diversas como: antibactericidas, antifungicos, antihelminticos, antitumorais, anticancererigenos,
antinflamatorios, antivirais, imunossupressores, inseticidas e herbicidas \cite{demain2009,jose2021}.
e segundo \citeonline{genilloud2017}, continuam sendo uma fonte relevante
para o isolamento de caracterização de compostos de interesse biotecnológicos, com o
emprego de metodologias modernas de investigação como a análise gênomica preditiva
podem continuar a fornecer substâncias de relevantes para mercado.

\subsection{\textit{Streptomyces}}
O gênero dos Streptomicetos é de grande relevância, pois foi a fonte para descobertas de importantes antibióticos 
como: estreptomicina, gentamicina, kanamicina, eritromicina e diversos outros \cite{demain2009}.
\citeonline{jose2021} descrevem que, do ano de 2014 a 2019 65\% do 549 compostos descobertos
de actinomicetos advinha de Streptomicetos, esse percentual demonstra que mesmo com o uso de técnicas
cada vez mais avançadas para busca de compostos, o gênero continua sendo um alvo interessante
para a descoberta de compostos. 



\subsection{\textit{Rhodococcus}}
\subsection{\textit{Kitastospora}}
\section{\textit{Bacillus}}

\section{Estudo genômico de MIB's}
\citeonline{ramirez2022} ressalta a relevância de bactérias para a descoberta de importantes
fármacos e propõe que organismos de fontes não convencionais como cavernas, fontes termais,
areas de alta salinidade, solos áridos, ocenos e mares continuem sendo estudados epecialmente
com tecnologias como metagênomica e mineração genômica pois podem ter um papel importante no 
combate de possíveis surtos de doeças como a SARS-COV2 e epidemias causadas por bactérias resistentes.

Em condições laboratoriais, muitos genes relacionados a síntese de
compostos bioativos são silênciados, limitando a produção a produção desses produtos, propondo
que o uso de eliciadores é necessário para expressão dos genes releacionados a produção desses 
compostos\cite{rutledge2015}. \citeonline{felicio2021} propõe o uma metodologia de eliciação para expressão, purificação
e caracterização desses compostos além de ressaltar que até 45\% dos compostos produzidos por
microorganismos são metabólitos secundários eliciados.

Através de Tecnologias modernas como a ferramenta ANTI-SMASH \cite{antismash} é possível predizer
genes putativos e \textit{clusters} gênicos relacionados a produção de metabólitos secundários
e de síntese ribossomal. Essa tecnologia de mineração \textit{in silico} permite prever redes
metabólicas e possíveis promotores da expressão desses compostos, principalmente por utilizar
bancos de dados produzidos a partir de outras ferramentas como BAGEL, NORINE e CLUSEAN \cite{bagel2,bagel3,norine,clusean}.
A incorporação de diversas ferramentas e banco de dados permite uma análise robusta 
e completa utilizando tecnologias do estado da arte da biologia computacional.

