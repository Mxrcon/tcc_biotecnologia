\chapter{Referencial Teórico}\label{cap:referenciais_teoricos}

\section{Metabolitos secundários}

O metabolismo celular bacteriano é o conjunto de processos bioquímicos anabólicos e catabólicos no qual
as células bacterianas produzem novos substâncias a partir de substrato ou outras substâncias, os produtos
dessas reações são conhecidos como metabólitos. Podendo ser classificados como primários ou secundários.
Os metabólitos primários são o conjunto de substâncias essenciais para a sobrevivência do organismo, relacionadas 
a produção de energia e as funções vitais da célula, já os secundários não estão relacionados a sobrevivência
da célula, mas sim sua perpetuação no ambiente utilizando estratégias de adaptação a situações adversas \cite{gokulan2014}.  

A maquinaria responsável pela produção desses compostos, normalmente está relacionada a aglomerados 
de genes biossintéticos (\textit{Biosyntetic Genes Cluster - BGC}). Esses aglomerados são estruturas gênicas
que contem dois ou mais genes agrupados contendo componentes da via biossintética para a produção de um metabólito, sendo capazes de produzir compostos
das seguintes classes: alcaloides, carboidratos, esteroides, lipídeos, peptídeos (com ou sem modificações pós-traducionais), policetídeos e
terpenoides \cite{medema2015}. 

Esses metabólitos possuem uma diversa gama de funções, seja como metodologia de "guerra
química" com outros microrganismos, mediadores de atividade mutualística entre espécies ou
simbiose química \cite{obrien2011}. Apesar de não serem considerados essenciais para a vida 
desses organismos \cite{demain2009} são de grande importância para sua dispersão e adaptação
em ambientes hostis e escassos de nutrientes. 

É importante ressaltar que a produção de metabólitos de atividade antimicrobiana, está relacionada
com a resistência a antimicrobianos, uma vez que, microrganismos produtores de substâncias antimicrobianas
precisam resistir a sua ação de forma a evitar o suicídio causado pelas suas próprias substâncias \cite{cundliffe2010avoidance}.

\section{Resistência a antimicrobianos}
Bactérias possuem diversos mecanismos para proteção contra agentes antimicrobianos como: desativação do fármaco, 
mutação no sítio de ligação do fármaco, expressão de bombas de efluxo e desvios metabólicos. Esses mecanismos, podem
estar associados a elementos genéticos móveis permitindo a transferência entre indivíduos da mesma espécie ou não \cite[p. 150]{Madigan2021}.

A resistência a antibióticos é uma questão emergente que está associada a mortalidade causada por patógenos bacterianos e sua solução
é complexa e permeia a necessidade de políticas públicas, vigilância e controle do uso de antibióticos, medidas de prevenção 
e o desenvolvimento de novas opções de tratamento \cite{frieri2017antibiotic}.  \citeonline{laxminarayan2013antibiotic} descrevem
extensamente a emergência da crise global no uso dos antibióticos, demarcando o problema não apenas como 
o uso inadequado pelos pacientes mas um grande conjunto que perpassa pelo enfraquecimento de sistemas
públicos de saúde, demora em diagnósticos e o uso de antibióticos como promotores de crescimento
na indústria agropecuária.

Para enfrentar esses problemas se faz necessária a readequação da metodologia de desenvolvimento
de fármacos, integrando o estado da arte da metabolômica, biologia sintética, genômica e química farmacêutica.
O uso de metodologias que integrem as informações advindas do genoma de bactérias resistentes 
com a criação de fármacos \cite{brown2016antibacterial}.

De acordo com \citeonline{cook2022past}, ainda há esperança na luta contra a crise dos antibióticos pois
 campos de estudo como a mineração genômica e biologia sintética aliadas de ferramentas poderosas como
a inteligência artificial combinadas com a reformulação de técnicas clássicas do desenvolvimento de fármacos
permitirão o desenvolvimento de tratamentos eficientes e diversos contra as patologias bacterianas. 

\section{Mineração genômica}
\citeonline{medema2021mining} descrevem a mineração genômica como uma forma de iluminar a química
altamente especializada da vida, uma ferramenta poderosa para o estudo do funcionamento de extensas
maquinarias genéticas para a produção de compostos com diversas funcionalidades. Esses estudo servem
como unificadores da ecologia e fisiologia além de servir como matéria prima para o desenvolvimento
de produtos biotecnológicos.

O estado da arte na mineração genômica tem criado bancos de dados gigantescos
com material descrevendo diversos tipos de genes relacionados a produção de metabólitos
de interesse, não apenas fornecendo informações ecológicas e filogenéticas, mas também oportunizando
o desenvolvimento de novas ferramentas capazes de auxiliar na descoberta de novos genes \cite{chevrette2021confluence}.

A mineração genômica como metodologia integrativa da bioinformática com outros campos
do conhecimento como a química e a fisiologia microbiana, permite que microrganismos com características
fenotípicas interessantes sejam extensamente estudados em todos os âmbitos da expressão e produção
de metabólitos. Inclusive predizendo propriedades desses metabólitos, sem depender de níveis de expressão,
tempo de crescimento ou da purificação. \cite{bauman2021genome, baltz2021genome}

\section{Actinomicetos}

Actinomicetos são um filo de microorganismos gram-positivos de alto conteúdo
guanina e citosina que contém as classes: \textit{Acidimicrobiia}, \textit{Actinobacteria}, 
\textit{Coriobacteriia}, \textit{Nitriliruptoria}, \textit{Rubrobacteria}, e \textit{Thermoleophilia}\cite{yadav2018}.
Dentre suas principais caracteristicas podemos ressaltar a presença de micélios
e a produção de hifas filamentosas \cite{chater2016}. Sua dispersão ambiental é enorme
e já foram isolados de ambientes diversos como: lagos salinos, mar profundo e solo \cite{flores2021,felicio2021,sapkota2020}.
Além da simbose com animais, fungos, insetos, línquens e plantas \cite{hei2021,van2017}.
A capacidade de se adaptar a diversos ambientes está intimamente relacionada com a capacidade
de produzir substâncias bioativas com funções igualmente diversas  \cite{van2020}

Essas bactérias foram uma fonte importante para o desenvolvimento de compostos de funções
diversas como: antibactericidas, antifungicos, antihelminticos, antitumorais, anticancererigenos,
antinflamatorios, antivirais, imunossupressores, inseticidas e herbicidas \cite{demain2009,jose2021}. 

64\% dos antibióticos derivados de produtos naturais foram obtidos a partir de actinomicetos filamentosos $($ especialmente do gênero \textit{Streptomyces} $)$,
especialmente durante a era de ouro dos antibióticos (1940-1960) sendo 20 utilizados clinicamente \cite{hutchings2019}.
Segundo \citeonline{genilloud2017}, continuam sendo uma fonte relevante
para o isolamento de caracterização de compostos de interesse biotecnológicos, e com o
emprego de metodologias modernas de investigação podem continuar a fornecer
substâncias relevantes para mercado. 


\subsection{\textit{Rhodococcus}}
O gênero \textit{Rhodococcus} contem actinomicetos de diversidade genômica e fisiológica,
contendo alguns membros patógenos para humanos, animais e plantas. Sua importância biotecnológica
é encontrada principalmente por conter algumas cepas com capacidade de degradar compostos orgânicos.
Devido seu grande tamanho gênomico (8.5-10Mb) esses microrganismos possuem grande diversidade genômica
de genes não essenciais, e através de eventos de recombinação, translocação e inserção, são capazes
de modificar seus próprios genomas e reservar genes codificantes de vias metabólicas diversas \cite{cappelletti2019}.

\textit{Rhodococcus} são os microorganismos mais adequados para o desenvolvimento de
tecnologias de remediação de ambientes por serem capazes de degradar poluentes persistentes e por
terem sido isolados de ambientes contaminados com hidrocarbonetos 
(inclusive em forma gasosa)\cite{kuyukina2019}. Sua resitência a intemperes como frio, calor, acidez,
salinidade, pode ser explorada para o desenvolvimentode biorremediadores de derramamento de derivados
de petróleo.

Além da capacidade de remediação biológica, podemos ressaltar o potencial de produção de diversas moléculas
como: biosulfactantes, biofloculantes, carotenoides, ácidos graxos poli-insaturados, poli-hidroxi-alcalóides e
triacil-glicerois \cite{cappelletti2020}. Os elementos de estrutura química complexa como os
carotenoides e os ácidos graxos poli-Insaturados são de grande interesse industrial pois sua síntese 
é complexa e custosa, o uso de microorganismos pode facilitar e reduzir os custos nesses processos.

Dentre as possibilidades para o uso biotecnológico de \textit{Rhodococcus} temos o uso como biofábricas 
para óleo, biocatálise em processos industriais e valorização de rejeitos \cite{alvarez2021,krivoruchko2019,anthony2019,chatterjee2020}.

\section{\textit{Brevibacillus brevis}}
\textit{Brevibacillus} (anteriormente \textit{Bacillus brevis}) é um gênero com grande potêncial para uso como organismo de expressão heteróloga
por ter crescimento rápido, baixa produção de proteases extracelulares e boa eficiência de transformação
por eletroporação, além disso diversos membros do gênero produzem substâncias com atividades 
larvicidas e antimicrobianas e tem grande importância agroecológica por sua relação mutualística
com plantas promovendo seu crescimento,as protegendo de patógenos e removendo metais pesados do solo
\cite{panda2014brevibacillus,ray2020brevibacillus}.  

\citeonline{yao2020available} ressaltam  capacidade prolífica de \textit{Brevibacillus} para expressão heteróloga
especialmente sua capacidade de produzir moléculas com eficiência ao ser mediada por promotores endógenos
com repetição em tandem de peptídeos sinal, sugerindo a importância do uso de estratégias eficazes de otimização do hospedeiro, do vetor,
do processo fermentativo e o estudo detalhado dos promotores do gênero através de estudos genômicos para melhoria desse modelo.

Exemplos importantes de metabólitos obtidos de \textit{Brevibacillus} temos os peptídeos antimicrobianos (\textit{Antimicrobial Pepitides} - AMP),
sendo esses classificados pela sua síntese ribossomal ou não, tendo diversos usos como o biocontrole em plantas, preservantes
para alimentos em prateleiras\cite{yang2018antimicrobial}. Além dos AMPS podemos citar a probdigiosina com atividade algicida e compostos ainda não
elucidados com grande atividade antiproliferativa \cite{zhang2022transcriptome,arumugam2018isolation}.
Além dos metabólitos, algumas vias bioquímicas dos \textit{Brevibacillus} são interessantes pela capacidade de
degradar Ácido Polilático (plástico biodegradável), a síntese de exopolissacarídeos e a 
biodegradação de polietileno \cite{yu2022comparison,yildiz2015genomic,hadad2005biodegradation,ali2022screening}.

A espécie \textit{Brevibacillus Brevis} contém indivíduos majoritariamente mesofílicos, e
sua distinção é baseada em similaridade genômica, sondagem molecular e análises 
quimiotaxonômicas \cite{ray2020brevibacillus}.  

\section{Estudo MIB's}
Em 2012 \citeonline{berdy2012thoughts} comentou a respeito do declínio no desenvolvimento de novos
fármacos, sendo eles resultantes de falhas humanas devido ao uso irresponsável de medicamentos,
falhas científicas devido a limitações técnicas e ambientes econômicos de regulação custosos e estritos
que limitam o desenvolvimento de novos medicamentos.
Após o fim da era de ouro dos antibióticos, um decréscimo dramático foi observado no nos níveis
de descoberta de novos fármacos, apesar disso, o desenvolvimento nas áreas de espectrometria de massas,
metabolômica, genômica e transcriptômica além do baixo custo para o sequenciamento de um genoma são
ferramentas importantes para o direcionamento no desenvolvimento de novos compostos derivados de produtos naturais \cite{katz2016natural}. 

O uso de ferramentas computacionais para a mineração 
de dados e predição de informações a partir de genomas é uma estratégia promissora por ser
eficiente economica e laboriosamente, e podem servir de estratégias guiadoras para o uso de outras 
ferramentas \cite{adamek2017mining}. \citeonline{trivella2018tripod} propõe o uso de um tripé para a identificação de novos produtos
naturais derivados de bactérias, sendo estes: o uso de mineração genômica, a manipulação de condições
de cultivo para eliciação da expressão de genes e a metabolômica baseada em espectrometria de massa. Com
essas ferramentas, os autores acreditam que a integração da genômica com técnicas de obtenção e purificação
de metabólitos serão as bases para o desenvolvimento de novos produtos farmacológicos pelos próximos anos.

\citeonline{ramirez2022} ressalta a relevância de bactérias para a descoberta de importantes
fármacos e propõe que organismos de fontes não convencionais como cavernas, fontes termais,
areas de alta salinidade, solos áridos, oceanos e mares continuem sendo estudados especialmente
com tecnologias como metagnômica e mineração genômica pois podem ter um papel importante no 
combate de possíveis surtos de doenças como a SARS-COV2 e epidemias causadas por bactérias resistentes.

Em condições laboratoriais, muitos genes relacionados a síntese de
compostos bioativos são silenciados, limitando a produção desses metabólitos,sendo necessário o uso de eliciadores para expressão dos genes relacionados a produção desses 
compostos\cite{rutledge2015}. \citeonline{felicio2021} propõe o uma metodologia de eliciação para expressão, purificação
e caracterização desses compostos além de ressaltar que até 45\% dos compostos produzidos por
microrganismos são metabólitos secundários eliciados.

A incorporação de diversas ferramentas e banco de dados permite uma análise robusta 
e completa utilizando tecnologias do estado da arte da biologia computacional. Através da ferramenta ANTI-SMASH \cite{antismash} podemos predizer
\textit{clusters} gênicos relacionados a produção de metabólitos secundários
e de síntese ribossomal. Essa tecnologia de mineração \textit{in silico} permite prever redes
metabólicas e possíveis promotores da expressão desses compostos, principalmente por utilizar
bancos de dados produzidos a partir de outras ferramentas como BAGEL, NORINE e CLUSEAN \cite{bagel2,bagel3,norine,clusean}.

