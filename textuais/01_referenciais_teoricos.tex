\chapter{Referenciais Teóricos}
\label{cap:referenciais_teoricos}

\section{Microoganismos de Interesse Biotecnológico (MIB)}

\subsection{Metabolitos secundários e a descoberta de fármacos}

\section{Actinomicetos}
Actinomicetos são um filo de microorganismos gram-positivos de alto conteúdo
guanina e citosina que contém as classes: Acidimicrobiia, Actinobacteria, 
Coriobacteriia, Nitriliruptoria, Rubrobacteria, e Thermoleophilia\cite{yadav2018}.
Dentre suas principais caracteristicas podemos ressaltar a presença de micélios
e a produção de hifas filamentosas \cite{chater2016}.
Essas bactérias foram uma fonte importante para o desenvolvimento de compostos de funções
diversas como: antibactericidas, antifungicos, antihelminticos, antitumorais, anticancererigenos,
antinflamatorios, antivirais, imunossupressores, inseticidas e herbicidas \cite{demain2009,jose2021}.
e segundo \citeonline{genilloud2017} continuam sendo uma fonte relevante
para o isolamento de caracterização de compostos de interesse biotecnológicos, com o
emprego de metodologias modernas de investigação como a análise gênomica preditiva
podem continuar a fornecer substâncias de relevantes para mercado.

\subsection{\textit{Streptomyces}}
O gênero dos Streptomicetos é de grande relevância, pois foi a fonte para descobertas de importantes antibióticos 
como: estreptomicina, gentamicina, kanamicina, eritromicina e diversos outros \cite{demain2009}
\citeonline{jose2021} descrevem que do ano de 2014 a 2019 65\% do 549 compostos descobertos
de actinomicetos advinha de Streptomicetos, esse percentual demonstra que mesmo com o uso de técnicas
cada vez mais avançadas para busca de compostos, o gênero continua sendo um alvo interessante
para a descoberta de compostos. 



\subsection{\textit{Rhodococcus}}
\subsection{\textit{Kitastospora}}
\section{\textit{Bacillus}}

\section{Estudo genômico de MIB's}

\citeonline revelam que em condições laboratoriais, muitos gemes relacionados a síntese de
compostos bioativos são silênciados, limitando a produção a produção desses produtos, propondo
que o uso de eliciadores é necessário para expressão dos genes releacionados a produção desses 
compostos.

Através de Tecnologias modernas como a ferramenta ANTI-SMASH \cite{antismash} é possível predizer
genes putativos e \textit{clusters} gênicos relacionados a produção de metabólitos secundários
e de síntese ribossomal. Essa tecnologia de mineração \textit{in silico} permite prever redes
metabólicas e possíveis promotores da expressão desses compostos, principalmente por utilizar
bancos de dados produzidos a partir de outras ferramentas como BAGEL\cite{bagel2,bagel3},
NORINE \cite{norine} e CLUSEAN \cite{clusean}. A incorporação de diversas ferramentas e banco de dados
permite uma análise robusta e completa utilizando tecnologias do estado da arte da biologia computacional.

