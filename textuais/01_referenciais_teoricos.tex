\chapter{Referenciais Teóricos}
\label{cap:referenciais_teoricos}

\section{Metabolitos secundários}

O metabolismo celular bacteriano é o conjunto de processos bioquímicos anabólicos e catabólicos no qual
as células bacterianas produzem novos substâncias a partir de substrato ou outras substâncias, os produtos
dessas reações são conhecidos como metabólitos. Podendo ser classificados como primários ou secundários, 
sendo os primários o conjunto de substâncias essenciais para a sobrevivência do organismo, relacionadas 
a produção de energia e as funções vitais da célula, já os secundários não estão relacionados a sobrevivência
da célula, mas sim sua perpetuação no ambiente utilizando estratégias de resistência a situações adversas \cite{gokulan2014}.  

A maquinaria responsável pela produção desses compostos, normalmente está relacionada a aglomerados 
de genes biossintéticos (\textit{Biosyntetic Genes Cluster - BGC}) que são dois ou mais genes 
agrupados codificam a via biosintética para a produção de um metabólito, sendo capazes de produzir compostos
das seguintes classes: alcalóides, carboidratos, esteroídes, lipídeos, peptídeos (com ou sem modificações pós-traducionais), policetídeos e
terpenóides \cite{medema2015}. 

Esses metabólitos possuem uma diversa gama de funções, seja como metodologia de "guerra
química" com outros microorganismos, mediadores de atividade mutualística entre espécies ou
simbiose química \cite{obrien2011}. Apesar de não serem considerados essenciais para a vida 
desses organismos \cite{demain2009} são de grande importância para sua dispersão e adaptação
em ambientes hostis e excassos de nutrientes. 

\section{Actinomicetos}

Actinomicetos são um filo de microorganismos gram-positivos de alto conteúdo
guanina e citosina que contém as classes: Acidimicrobiia, Actinobacteria, 
Coriobacteriia, Nitriliruptoria, Rubrobacteria, e Thermoleophilia\cite{yadav2018}.
Dentre suas principais caracteristicas podemos ressaltar a presença de micélios
e a produção de hifas filamentosas \cite{chater2016}. Sua dispersão ambiental é enorme
e já foram isolados de ambientes diversos como: lagos salinos, mar profundo e solo \cite{flores2021,felicio2021,sapkota2020}.
Além da simbose com animais, fungos, insetos, línquens e plantas \cite{hei2021,van2017}.
A capacidade de se adaptar a diversos ambientes está intimamente relacionada com a capacidade
de produzir substâncias bioativas com funções igualmente diversas  \cite{van2020}

Essas bactérias foram uma fonte importante para o desenvolvimento de compostos de funções
diversas como: antibactericidas, antifungicos, antihelminticos, antitumorais, anticancererigenos,
antinflamatorios, antivirais, imunossupressores, inseticidas e herbicidas \cite{demain2009,jose2021}.
Segundo \citeonline{genilloud2017}, continuam sendo uma fonte relevante
para o isolamento de caracterização de compostos de interesse biotecnológicos, e com o
emprego de metodologias modernas de investigação podem continuar a fornecer
substâncias relevantes para mercado.

\subsection{\textit{Streptomyces}}
O gênero dos estreptomicetos é de grande relevância, pois foi a fonte para descobertas de importantes antibióticos 
como: estreptomicina, gentamicina, kanamicina, eritromicina e diversos outros \cite{demain2009}.
\citeonline{jose2021} descrevem que, do ano de 2014 a 2019 65\% do 549 compostos descobertos
de actinomicetos advinha de bactérias do gênero Streptomyces, ressaltando a importância do gênero.


\subsection{\textit{Rhodococcus}}
o gênero \textit{Rhodococcus} contem actinomicetos de diversidade genômica e fisiológica,
contendo alguns membros patógenos para humanos, animais e plantas. Sua importância biotecnológica
é encontrada principalmente por conter algumas cepas com capacidade de degradar compostos orgânicos.
Devido seu grande tamanho gênomico (8.5-10Mb) esses microorganismos possuem grande liberdade
para modifcar seu genoma com recombinações, translocações e inserções, contendo diversas vias catabólicas,
e mantendo multiplas funções metabólicas \cite{cappelletti2019}.

\textit{Rhodococcus} são os microorganismos mais adequados para o desenvolvimento de
tecnologias de remediação de ambientes por serem capazes de degradar poluentes persistentes e por
terem sido isolados de ambientes contaminados com hidrocarbonetos 
(inclusive em forma gasosa)\cite{kuyukina2019}. Sua resitência a intemperes como frio, calor, acidez,
salinidade, pode ser explorada para o desenvolvimentode biorremediadores de derramamento de derivados
de petróleo.

Além da capacidade de remediação biológica, podemos ressaltar o potencial de produção de diversas moléculas
como: Biosulfactantes, Biofloculantes, Carotenoides, Acidos Graxos Poli-Insaturados, Poli-Hidroxi-Alcalóides e
Triacil-Glicerois \cite{cappelletti2020}. Essas estruturas especialmente as de estruturas complexas como os
Carotenoides e os Ácidos graxos Poli-Insaturados são de grande interesse industrial pois sua síntese 
é complexa e custosa, o uso de microorganismos pode facilitar e reduzir os custos nesses processos.

Dentre as possibilidades para o uso biotecnológico de \textit{Rhodococcus} temos o uso como biofábricas 
para óleo, biocatálise em processos industriais e valorização de rejeitos \cite{alvarez2021,krivoruchko2019,anthony2019,chatterjee2020}.

\subsection{\textit{Kitastospora}}
\section{\textit{Bacillus}}

\section{Estudo genômico de MIB's}
- Nesse tópico podes iniciar abordando a questão do desinteresse em prospectar novas moleculas por conta dos processos padrões serem custosos, e que isso levou ao desinteresse da industria. 
- No  entanto com o advento de novas tecnologias (escreve um pouco de cada), estão sendo retomadas a exploração pelo potencial biossintetico de microrg.
- Que antes do sequenciamento do genoma pouco se sabia sobre o potencial biossintetico das bacterias
- Depois inicia sobre a importancia das analises genomicas, cita estudos que mostram o amplo conteudo de genes biossinteticos
- Podes abordar sobre cada ferramenta que vais utilizar, apesar de que eu acho que metodologia não deve ter na introdução.


\citeonline{ramirez2022} ressalta a relevância de bactérias para a descoberta de importantes
fármacos e propõe que organismos de fontes não convencionais como cavernas, fontes termais,
areas de alta salinidade, solos áridos, ocenos e mares continuem sendo estudados epecialmente
com tecnologias como metagênomica e mineração genômica pois podem ter um papel importante no 
combate de possíveis surtos de doeças como a SARS-COV2 e epidemias causadas por bactérias resistentes.

Em condições laboratoriais, muitos genes relacionados a síntese de
compostos bioativos são silênciados, limitando a produção a produção desses produtos, propondo
que o uso de eliciadores é necessário para expressão dos genes releacionados a produção desses 
compostos\cite{rutledge2015}. \citeonline{felicio2021} propõe o uma metodologia de eliciação para expressão, purificação
e caracterização desses compostos além de ressaltar que até 45\% dos compostos produzidos por
microorganismos são metabólitos secundários eliciados.

Através de Tecnologias modernas como a ferramenta ANTI-SMASH \cite{antismash} é possível predizer
genes putativos e \textit{clusters} gênicos relacionados a produção de metabólitos secundários
e de síntese ribossomal. Essa tecnologia de mineração \textit{in silico} permite prever redes
metabólicas e possíveis promotores da expressão desses compostos, principalmente por utilizar
bancos de dados produzidos a partir de outras ferramentas como BAGEL, NORINE e CLUSEAN \cite{bagel2,bagel3,norine,clusean}.
A incorporação de diversas ferramentas e banco de dados permite uma análise robusta 
e completa utilizando tecnologias do estado da arte da biologia computacional.

