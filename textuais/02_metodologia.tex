\chapter{Metodologia}
\section{Isolamento e obtenção das amostras}
As amostras bacterianas utilizadas neste estudo foram obtidas do banco de isolados de outro
trabalho, disponibilizados pelo Centro de Genômica e Biologia de sistemas. As quais foram
obtidas apartir da cultura em meio Ágar Amido Proteina-M (HI-MEDIA) para isolamento de actinomicetos.

\section{Extração de DNA e identificação das espécies}
Os microorganismos foram cultivados em meio TSB e centrifugados a 13000rpm por 15 minutos,
os \textit{pellets} foram coletados e utilizados para extração usando o kit DNEasy e método
adaptado de extração usando  fenol-clorofórmio-álcool-isoamil. O DNA gênomico extraído
foi quantificado utilizando espectrofotometria com NanoDrop(Thermo Fisher) e visualizado
em gel de agarose 1,5\% com brometo de etídio. 
Foi utilizada PCR com os primers %TODO
para amplificação do RNA ribossomal 16s, os amplicons foram sequênciados utilizando 3500 
genetic analyzer (Applied Biosystems). Os fragmentos \textit{foward} e \textit{reverse}
foram alinhados utilizando BioEdit(TODO) e as \textit{contigs} produzidas foram
alinhadas ao banco de dados de RNA ribossomal do NCBI com a ferramenta BLAST e as espécies
dos organismos foram determinadas utilizando homologia.  

\section{Identificação de perfil clonal}
Para análise de perfil clonal foi utilizada PCR para sequências palindrômicas
extragênicas repetidas(\textit{REP}) com os primers F e R, os produtos amplificados
foram submetidos a eletroforese em gel de agarose 1.5\% com adição de 0.5\% de
brometo de etídio por 1 hora e 30 minutos, com corrente de 70V 70mAh e 70W.

O gel foi fotografado e o tamanho dos fragmentos foi analisado usando a ferramenta
Gel Analyzer e uma tabela com os dados foi utilizada para cálculo de \textit{clusters}
dos isolados e criação de um dendograma com o uso de um \textit{script} na
linguagem R.

\section{Sequênciamento}
Foram selecionados 4 microorganismos de espécies diferentes, sendo 3 Actionomicetos
e uma bactéria do gênero \textit{Bacillus} que foi capaz de inibir o crescimento de
outros microorganismos durante o isolamento. As bibliotecas foram preparadas utilizando
as recomendações do fabricante e sequênciadas com a plataforma %TODO

\section{Mineração de Informações Genômicas}
Após o Sequênciamento as amostras foram submetidas ao pipeline Bactopia, o qual filtrou
as leituras, montou e anotou o genoma. Após isso, foram utilizadas as Bactopia \textit{tools}
para análise de resistência, genes patogênicas, genes de produção de compostos, clusters
gênicos e elementos moveis. 
Foram utilizados os softwares GoFeat, Brig e R para criação de figuras a partir dos dados gerados.
