\chapter{Metodologia}
\section{Seleção de amostras}

Diante da importância biotecnologia dos microrganismos citado anteriormente, 
para a exploração genômica foram selecionadas duas cepas do parque estadual Utinga - Belém, PA gentilmente disponibilizadas pelo Centro de Gênomica e Biologia de Sistemas.
Incluindo uma actinobactéria do gênero \textit{Rhodococcus} (ACT\_016) e e uma bactéria do filo \textit{Firmicutes}: \textit{Brevibacillus brevis}(FIR\_094).
Essa amostras foram previamente identificadas utilizando sequênciamento do gene de RNA ribossomal 16s
utilizando os primers universais 8F: 5'-AGAGTTTGATCATGGCTCAG-3' e 1492R: 5'-CGGTTACCTTGTTACGACTT-3' com o sequenciador 
ABI Prism 3500 Genetic Analyzer (Applied BioSystems). Posteriormente as espécies foram preditas utilizando
homologia baseada no alinhamento contra o banco de dados de RNA ribossomal do NCBI utilizando a ferramenta
blast.

\section{Extração de DNA e sequenciamento}
As amostras foram cultivadas em meio Tryptone Soy Broth (TSB) por 48 horas á 28 graus, e
seu DNA foi extraído utilizando o kit HiPureA Multi-sample DNA Purification Kit(HI-MEDIA) seguindo as orientações
do fabricante. O DNA foi quantificado usando quantificador Qubit(Thermo Fisher) e sua integridade foi 
avaliada por eletroforese em gel de agarose 1\% complementado com brometo de etídeo 0.5\%.
As bibliotecas foram preparadas utilizando o protocolo do fabricante e sequenciadas no equipamento
Ion GeneStudio S5 Plus (Thermo Fisher)

\section{Análise Genômica}
O pipeline Bactopia \cite{Bactopia}, filtrou as leituras, montou e anotou o genoma automaticamente.
Paralelamente, as amostras foram filtradas manualmente utilizando a ferramenta Trimmomatic \cite{bolger2014trimmomatic}
utilizando os seguintes parâmetros: "LEADING:3 TRAILING:30 SLIDINGWINDOW:4:15 MINLEN:36"
e foram realizadas montagens manuais utilizando o software Shovill $($ não publicado $)$ com os 
montadores SKESA\cite{souvorov2018skesa} e SPADES\cite{bankevich2012spades} utilizando
parâmetros automáticos e kmers específicos (21,33,55,77,99,127).
As melhores montagens foram selecionadas após visualização da qualidade no software QUAST \cite{gurevich2013quast}. 

Após isso as melhores montagens foram submetidas ao programa KRAKEN \cite{wood2019improved} para determinar a pureza
das montagens e predição de espécies. Posteriormente os genomas foram montados em um único cromossomo
utilizando o software RAGOUT\cite{kolmogorov2014ragout} utilizando genomas de referência próximos a espécie predita pelo KRAKEN, 
a pureza das montagens foi verificada utilizando o software BUSCO\cite{simao2015busco}.

Finalmente os genomas foram anotados utilizando o software
PROKKA \cite{seemann2014prokka} , seus genes de resistência a antibióticos foram preditos utilizando o software armfinder e seus clusters de metabólitos
secundários foram preditos utilizando a ferramenta ANTISMASH\cite{blin2021antismash}.

Foram utilizados os softwares fastqc\cite{andrews2019fastqc} e Artemis\cite{carver2012artemis} para criação de figuras a partir dos dados gerados.
