\chapter{Metodologia}
\section{Seleção de amostras}
Foram selecionados 4 microorganismos de espécies diferentes do banco de amostras ambientais provenientes
do parque estadual Utinga - Belém, PA gentilmente disponibilizadas pelo Centro de Gênomica e Biologia de Sistemas.
Incluindo três Actinobacterias: \textit{Kitasatospora sp.},\textit{Rhodococcus sp.} e \textit{Streptomyces sp.}
e uma bactéria do filo \textit{Firmicutes}: \textit{Brevibacillus brevis}.
Essa amostras foram previamente identificadas utilizando sequênciamento do gene de RNA ribossomal 16s
utilizando os primers universais 8F: 5'-AGAGTTTGATCATGGCTCAG-3' e 1492R: 5'-CGGTTACCTTGTTACGACTT-3' com o sequenciador 
ABI Prism 3500 Genetic Analyzer (Applied BioSystems). Posteriormente as espécies foram preditas utilizando
homologia baseada no alinhamento contra o banco de dados de RNA ribossomal do NCBI utilizando a ferramenta
blast.

\section{Extração de DNA}
As amostras foram cultivadas em meio Tryptone Soy Broth (TSB) por 48 horas á 28 graus, e em
seu DNA foi extraído utilizando o kit HiPureA Multi-sample DNA Purification Kit(HI-MEDIA) seguindo as orientações
do fabricante. O DNA foi quantificado usando quantificador Quibit(TODO) e sua intigridade foi 
avaliada por eletroforese em gel de agarose 1\% complementado com brometo de estídio 0.5\%.

\section{Sequênciamento e análise genômica}
As bibliotecas foram preparadas utilizando o protocolo do fabricante e sequênciadas no equipamento
Ion GeneStudio S5 Plus (Thermo Fisher)
Após o sequênciamento as amostras foram submetidas ao pipeline Bactopia, o qual filtrou
as leituras, montou e anotou o genoma. Após isso, foram utilizadas as ferramentas do Bactopia 
para análise de resistência, genes patogênicos, genes de produção de compostos, clusters
gênicos e elementos moveis. 
Foram utilizados os softwares GoFeat,Anti-Smash versão 6, BRIG e R para criação de figuras a partir dos dados gerados.
