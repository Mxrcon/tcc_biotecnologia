\chapter{Metodologia}
\section{Seleção de amostras}
Foram selecionados 2 microorganismos de espécies diferentes do banco de amostras ambientais provenientes
do parque estadual Utinga - Belém, PA gentilmente disponibilizadas pelo Centro de Gênomica e Biologia de Sistemas.
Incluindo uma actinobactéria do gênero \textit{Rhodococcus} (ACT016) e e uma bactéria do filo \textit{Firmicutes}: \textit{Brevibacillus brevis}(ACT094).
Essa amostras foram previamente identificadas utilizando sequênciamento do gene de RNA ribossomal 16s
utilizando os primers universais 8F: 5'-AGAGTTTGATCATGGCTCAG-3' e 1492R: 5'-CGGTTACCTTGTTACGACTT-3' com o sequenciador 
ABI Prism 3500 Genetic Analyzer (Applied BioSystems). Posteriormente as espécies foram preditas utilizando
homologia baseada no alinhamento contra o banco de dados de RNA ribossomal do NCBI utilizando a ferramenta
blast.

\section{Extração de DNA e sequenciamento}
As amostras foram cultivadas em meio Tryptone Soy Broth (TSB) por 48 horas á 28 graus, e
seu DNA foi extraído utilizando o kit HiPureA Multi-sample DNA Purification Kit(HI-MEDIA) seguindo as orientações
do fabricante. O DNA foi quantificado usando quantificador Qubit(Thermo Fisher) e sua integridade foi 
avaliada por eletroforese em gel de agarose 1\% complementado com brometo de etídeo 0.5\%.
As bibliotecas foram preparadas utilizando o protocolo do fabricante e sequenciadas no equipamento
Ion GeneStudio S5 Plus (Thermo Fisher)

\section{Análise Genômica}
O pipeline Bactopia, filtrou as leituras, montou e anotou o genoma automaticamente.
Paralelamente, as amostras foram filtradas manualmente utilizando a ferramenta Trimmomatic
utilizando os seguintes parâmetros: "LEADING:3 TRAILING:30 SLIDINGWINDOW:4:15 MINLEN:36"
e foram realizadas montagens manuais utilizando o software Shovill com os montadores SKESA e SPADES utilizando
parâmetros automáticos e kmers específicos (21,33,55,77,99,127).
As melhores montagens foram selecionadas após visualização da qualidade no software QUAST. 
Após isso as melhores montagens foram submetidas ao programa KRAKEN para determinar a pureza
das montagens e predição de espécies. Posteriormente os genomas foram montados em um único cromossomo
utilizando o software RAGOUT utilizando genomas de referência proximos a espécie predita pelo KRAKEN, 
a pureza das montagens foi verificada utilizando o software BUSCO.

Finalmente os genomas foram anotados utilizando o software
PROKKA, seus genes de resistência a antibióticos foram preditos utilizando o software armfinder e seus clusters de metabólitos
secundários foram preditos utilizando a ferramenta ANTISMASH.

Foram utilizados os softwares fastqc e Artemis para criação de figuras a partir dos dados gerados.
