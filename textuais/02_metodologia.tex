\chapter{Metodologia}
\section{Seleção de amostras}
Foram selecionados 2 microorganismos de espécies diferentes do banco de amostras ambientais provenientes
do parque estadual Utinga - Belém, PA gentilmente disponibilizadas pelo Centro de Gênomica e Biologia de Sistemas.
Incluindo uma actinobactéria do gênero \textit{Rhodococcus} (ACT016) e e uma bactéria do filo \textit{Firmicutes}: \textit{Brevibacillus brevis}(ACT094).
Essa amostras foram previamente identificadas utilizando sequênciamento do gene de RNA ribossomal 16s
utilizando os primers universais 8F: 5'-AGAGTTTGATCATGGCTCAG-3' e 1492R: 5'-CGGTTACCTTGTTACGACTT-3' com o sequenciador 
ABI Prism 3500 Genetic Analyzer (Applied BioSystems). Posteriormente as espécies foram preditas utilizando
homologia baseada no alinhamento contra o banco de dados de RNA ribossomal do NCBI utilizando a ferramenta
blast.

\section{Extração de DNA e sequenciamento}
As amostras foram cultivadas em meio Tryptone Soy Broth (TSB) por 48 horas á 28 graus, e
seu DNA foi extraído utilizando o kit HiPureA Multi-sample DNA Purification Kit(HI-MEDIA) seguindo as orientações
do fabricante. O DNA foi quantificado usando quantificador Qubit(Thermo Fisher) e sua integridade foi 
avaliada por eletroforese em gel de agarose 1\% complementado com brometo de etídeo 0.5\%.
As bibliotecas foram preparadas utilizando o protocolo do fabricante e sequenciadas no equipamento
Ion GeneStudio S5 Plus (Thermo Fisher)

\section{Análise Genômica}
O pipeline Bactopia, filtrou as leituras, montou e anotou o genoma automaticamente.
Paralelamente, as amostras foram filtradas manualmente utilizando a ferramenta Trimmomatic
e foram realizadas montagens manuais utilizando o software Shovill com os montadores SKESA e SPADES.
As melhores montagens foram selecionadas após visualização da qualidade no software QUAST. 
Após isso as melhores montagens foram submetidas ao programa KRAKEN e BUSCO para determinar a pureza
das montagens e predição de espécies. Posteriormente os genomas foram montados em um único cromossomo
utilizando o software RAGOUT utilizando genomas de referência, finalmente os genomas foram anotados utilizando o software
PROKKA, seus genes de resistência a antibióticos foram preditos utilizando o ARIBA e seus clusters de metabólitos
secundários foram preditos utilizando a ferramenta ANTISMASH.

Foi gerada uma árvore filogenética utilizando os 50 hits do gene de RNA ribossomal 16s, alinhados
na ferramenta MAFT produzindo uma árvore filogenética por máxima verossimilhança com bootstrap 1000 usando RAXML.

Foram utilizados os softwares GoFeat, fastqc e Artemis para criação de figuras a partir dos dados gerados.
