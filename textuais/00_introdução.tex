\chapter{Introdução}
\label{cap:introducao}

\section{Contexto}


- Necessidade de novos Compostos

- Uso de Biotecnologia para solução de problemas industriais
A biotecnologia 

Os ambientes amazônicos são um reservatório de biodiversidade muito importantes,
a riqueza de espécies e o uso da biotecnologia como ferramenta para a solução de problemas relacionados
a saúde humana e animal, industriais eu patrimônio genético são fontes interessantes para o
desenvolvimento sustentável baseado no uso de tecnologia de ponta para a formulação de novas tecnologias.
micro-organismos do solo amazônico podem ser a fonte de novos fármacos para doenças já conhecidas,
a cura para doenças emergentes, biofábricas para novos processos industriais e biorremediadores
de impactos ambientais.

\section{Justificativa}
Bactérias ambientais são interessantes alvos para a descoberta de compostos
de relevância biotecnológica, especialmente como solução para os crescentes níveis
de resistência a antimicrobianos encontrados em microorganismos patogênicos.
A caracterização genômica e prospecção de genes de interesse desses microorganismos,
especialmente do ambiente amazônico, são passos importantes
em busca de genes para produção de compostos de potencial farmacológico e industrial.

\chapter{Objetivos}

\section{Objetivo Geral}

Predizer o potencial biotecnológico de bactérias ambientais utilizando 
ferramentas \textit{in silico} 

\section{Objetivos Específicos}
\begin{enumerate}
    \item Caracterizar os organismos sequênciados utilizando seus genomas
    \item Predizer as características metabólicas dos organismos
    \item Analisar os microorganismos quanto ao potêncial de produção de compostos de interesse biotécnológico
\end{enumerate}





