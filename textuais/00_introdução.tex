\chapter{Introdução}
\label{cap:introducao}

\section{Contexto}



\section{Justificativa}


\section{Objetivos}

\subsection{Objetivo Geral}

Identificar a produção de substâncias antimicrobianas em Actinomicetos

\subsection{Objetivos Específicos}
\begin{enumerate}
    \item alguma coisa a ver com identificação das amostras %TODO
    \item alguma coisa a ver com perfil clonal %TODO
    \item alguma coisa a ver com subst. %TODO
\end{enumerate}

\section{Metodologia}
\subsection{Coleta das amostras}
\subsection{REP}
\begin{enumerate}
    \item Primers%TODO
    \item Gel analyzer e Rscript%TODO
    \item Análise do dendograma%TODO
\end{enumerate}

\subsection{Teste de bioatividade}

\begin{enumerate}
    \item Centrifugação%TODO
    As amostras foram centrifugadas a 8000rpm por 5 minutos para separação dos metabólitos
    \item Amostras controle%TODO
    \textit{Bacilus subtilis, Staphylococcus Aureus, Echerichia coli}
    \item Antibiótico controle positivo e meio controle negativo%TODO
    \item Medição dos halos e tratamento estatístico
\end{enumerate}



\section{Estrutura do Trabalho}

% ---
% Capitulo com exemplos de comandos inseridos de arquivo externo 
% ---
\include{abntex2-modelo-include-comandos}
% ---

