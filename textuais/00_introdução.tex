\chapter{Introdução}
\label{cap:introducao}

\section{Contexto}
Actinomicetos são um filo de microorganismos gram-positivos de alto conteúdo
guanina e citosina que contém as classes: Acidimicrobiia, Actinobacteria, 
Coriobacteriia, Nitriliruptoria, Rubrobacteria, e Thermoleophilia\cite{yadav2018}.
Dentre suas principais caracteristicas podemos ressaltar a presença de micélios
e a produção de hifas filamentosas \cite{chater2016}.

Actinobactérias foram uma grande fonte para o desenvolvimento de diversos
compostos como:%TODO
 e segundo \citeonline{genilloud2017} continuam sendo uma fonte relevante
para o isolamento de caracterização de compostos de interesse biotecnológicos, com o
emprego de metodologias modernas de investigação podem continuar a fornecer 
substâncias de impacto mercadológico.

\section{Justificativa}
Bactérias ambientais são interessantes alvos para a descoberta de compostos
de relevância biotecnológica, especialmente como solução para os crescentes níveis
de resistência a antimicrobianos encontrados em microorganismos patogênicos.
O isolamento desses microorganismos, especialmente do ambiente amazônico,
é um passo importante em busca de compostos de potencial farmacológico e industrial.

\chapter{Objetivos}

\section{Objetivo Geral}

Determinar a produção de compostos antimicrobianos de actinomicetos
provenientes do solo do parque do utinga

\section{Objetivos Específicos}
\begin{enumerate}
    \item Identificar e caracterizar os organismos em nível de espécie
    \item Determinar o perfil clonal e a distância filogenética entre os isolados
    \item Categorizar os microorganismos quanto a produção de compostos bactericidas.
\end{enumerate}





