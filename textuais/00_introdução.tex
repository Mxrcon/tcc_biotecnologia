\chapter{Introdução}
\label{cap:introducao}

\section{Contexto}

\section{Justificativa}

\section{Objetivos}

\subsection{Objetivo Geral}

\subsection{Objetivos Específicos}

\section{Metodologia}

\subsection{Obtenção das amostras}

Cepas \textit{wild type} de \textit{Coryneacterium pseudotuberculosis biovar equi} foram gentilmente cedidas pelo Prof. Dr. Vasco Azevedo 
da Universidade Federal de Minas Gerais, estas cepas são identificadas pelas sigas MB122, MB154, MB302 e MB336, e seus genomas estão
depositados no banco de dados NCBI sob os códigos de acesso: TODO. As amostras foram isoladas, genotipadas e sequenciadas em um estudo anterior \cite{barauna2017}.



\subsection{Determinação da sensibilidade a antibióticos}

A sensibilidade aos antibióticos enroflaxacina e rifampicina foram determinadas utilizando teste de concentração inibitória mínima de acordo com o descrito nas
normas M100 do CLSI de 2020 \cite{clsi2020} e MD45 de 2015 \cite{clsi2015}. 

\subsection{Indução de resistência}
\subsection{Sequênciamento das amostras}
\subsection{Chamada, predição e avaliação dos variantes}

\section{Estrutura do Trabalho}

% ---
% Capitulo com exemplos de comandos inseridos de arquivo externo 
% ---
\include{abntex2-modelo-include-comandos}
% ---

