\chapter{Introdução}
\label{cap:introducao}

\section{Contexto}

A OMS relata uma crise na saúde pública derivada do crescente número
de microrganismos resistentes e representam uma grande ameaça global, que somente
pode ser resolvida se todos aqueles envolvidos no controle e na distribuição
dos medicamentos antimicrobianos cumprissem com suas responsabilidades e deveres 
quanto a esse problema \cite{talebi2019world}.

A exploração moderna de genes vias metabólicas sileciosa em microrganismos
através de técnicas do estado da arte em genômica e metabolômica, podem
revelar novos tesouros metabólicos para solucionar problemas como a crise
dos antibióticos \cite{zhang2021glossary}.

Os ambientes amazônicos são um reservatório de biodiversidade muito importantes,
a riqueza de espécies e o uso da biotecnologia como ferramenta para a solução de problemas industriais  e relacionados
a saúde humana e animal, seu patrimônio genético é fonte interessante para o
desenvolvimento sustentável baseado no uso de tecnologia de ponta para a formulação de novas tecnologias.
micro-organismos do solo amazônico podem ser a fonte de novos fármacos para doenças já conhecidas,
a cura para doenças emergentes, biofábricas para novos processos industriais e biorremediadores
de impactos ambientais.

\section{Justificativa}
Bactérias ambientais são interessantes alvos para a descoberta de compostos
de relevância biotecnológica, especialmente como solução para os crescentes níveis
de resistência a antimicrobianos encontrados em microrganismos patogênicos.
A caracterização genômica e prospecção de genes de interesse desses microrganismos,
especialmente do ambiente amazônico, são passos importantes
em busca de genes para produção de compostos de potencial farmacológico e industrial.

\chapter{Objetivos}

\section{Objetivo Geral}

Identificar agrupamentos de genes biossintéticos 
em bactérias de filos com potencial biotecnológico isoladas do Parque Estadual Utinga.

\section{Objetivos Específicos}
\begin{enumerate}
    \item Sequenciar o genoma de dois isolados de interesse;
    \item Montar, anotar e predizer os agrupamentos de genes biossintéticos;
    \item Avaliar o potencial biotecnológico dos isolados.
\end{enumerate}





