\chapter{Introdução}
\label{cap:introducao}

\section{Contexto}



\section{Justificativa}


\section{Objetivos}

\subsection{Objetivo Geral}

Identificar a produção de substâncias antimicrobianas em Actinomicetos

\subsection{Objetivos Específicos}
\begin{enumerate}
    \item alguma coisa a ver com identificação das amostras %TODO
    \item alguma coisa a ver com perfil clonal %TODO
    \item alguma coisa a ver com subst. %TODO
\end{enumerate}

\section{Metodologia}
\subsection{Isolamento e obtenção das amostras}
As amostras bacterianas utilizadas neste estudo foram obtidas do banco de isolados de outro
trabalho, disponibilizados pelo Centro de Genômica e Biologia de sistemas. As quais foram
obtidas apartir da cultura em meio Proteina-P (HI-MEDIA) para isolamento de actinomicetos,
o DNA desses isolados foi extraído utilizando o \textit{kit DNEasy} (QIAGEN) e
método \textit{in house} com fenol-clorofórmio-álcool-isoamil.

\subsection{Identificação de perfil clonal}
Para análise de perfil clonal foi utilizada PCR para sequências palindrômicas
%TODO
extragênicas repetidas(\textit{REP}) com os primers F e R, os produtos amplificados
foram submetidos a eletroforese em gel de agarose 1.5\% com adição de 0.5\% de
brometo de etídio por 1 hora e 30 minutos, com corrente de 70V 70mAh e 70W.

O gel foi fotografado e o tamanho dos fragmentos foi analisado usando a ferramenta
Gel Analyzer e uma tabela com os dados foi utilizada para cálculo de \textit{clusters}
dos isolados e criação de um dendograma com o uso de um \textit{script} na
linguagem R.

\subsection{Teste de bioatividade}
\subsubsection{Obtenção dos extratos}
Isolados que durante o isolamento inibiam o crescimento de outras amostras e outros 37
com perfis clonais diferentes foram cultivados em meio Ágar TSB para
reativação. Após isso, foram inoculados em Caldo TSB de 1 a 5 dias até turvação completa
do meio, esse caldo foi centrifugado a 9000$G$ por 10 minutos e o sobrenadante foi coletado
e filtrado em filtro 0.22$\mu$m para uso no mesmo dia.
\subsubsection{Teste de difusão em poço}
Foi utilizado teste de difusão em poço para determinação de atividade antimicrobiana,
para isso foi utilizado meio Müeller-Hinton Ágar, com furos de 10mm de diâmetro.
Em cada furo foi depositado $50\mu$L de extrato, as cepas teste utilizadas foram:
\textit{Bacillus Substilis ATCC 6633}, \textit{Corynebacterium Fimi NTC5}, \textit{Staphylococcus Aureus ATCC 25922}
e um isolado resistente de \textit{Escherichia Coli} obtido em estudo anterior \cite{dhara2019}.
Para cada teste, foi preparado inóculo em solução salina 1\% com ajuste para 0.5
na escala McFarland, essa solução foi uniformemente espalhada com uso de swab estéril.

Como controles positivos foram utilizados discos de antibióticos com
eficácia para cada uma das cepas controle, conforme descrito nas normas de controle e
análise para testes de substâncias antimicrobianas\cite{clsi2020}.

\begin{table}[!htb]
    \caption{Lista de antibióticos utilizados como controle para os testes de bioatividade}
    \label{tab:lista_antibioticos}
    \centering
    \begin{tabular}{ll}
    \toprule
    \textbf{Bactéria}                           & \textbf{Antibiótico controle} \\
    \rowcolor[HTML]{F3F3F3}
    \textit{Escherichia Coli ATCC 292} & Impinem               \\
    \rowcolor[HTML]{DBDEDE} 
    \textit{Staphylococcus aureus}     & Impinem               \\
    \rowcolor[HTML]{F3F3F3}
    \textit{Corynebacterium fimi}      & ?                     \\
    \rowcolor[HTML]{DBDEDE} 
    \textit{Escherichai Coli APC43A}   & ?                     \\
    \bottomrule \\
    \end{tabular}
    \begin{small}\textbf{Fonte: Autor}\end{small} 
\end{table}


Os halos de inibição foram medidos após o crecimento de 24 horas a $30$\textdegree$C$
e análisados estatisticamente utilizando teste de $\rho$ com
o uso da ferramenta R.

\section{Estrutura do Trabalho}

% ---
% Capitulo com exemplos de comandos inseridos de arquivo externo 
% ---
\include{abntex2-modelo-include-comandos}
% ---

