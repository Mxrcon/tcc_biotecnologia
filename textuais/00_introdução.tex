\chapter{Introdução}
\label{cap:introducao}

\section{Contexto}

\section{Justificativa}

\section{Objetivos}

\subsection{Objetivo Geral}

\subsection{Objetivos Específicos}

\section{Metodologia}

\subsection{Obtenção das amostras}

Cepas \textit{wild type} de \textit{Coryneacterium pseudotuberculosis biovar equi} foram gentilmente cedidas pelo Prof. Dr. Vasco Azevedo 
da Universidade Federal de Minas Gerais, estas cepas são identificadas pelas sigas MB122, MB154, MB302 e MB336, e seus genomas estão
depositados no banco de dados NCBI sob os códigos de acesso: TODO. As amostras foram isoladas, genotipadas e sequenciadas em um estudo anterior \cite{barauna2017}.



\subsection{Determinação da sensibilidade a antibióticos}

A sensibilidade aos antibióticos enroflaxacina e rifampicina foram determinadas utilizando teste de concentração inibitória mínima em microdilução em 
caldo mueller hinton de acordo com o descrito nas
normas M100 e MD45 do CLSI \cite{clsi2020, clsi2015}. Utilizando pré inóculo em meio \textit{Brain Heart Infusion} adicionado de \textit{tween} 80 0.2\%
e ajustado para 0.5 na escala \textit{MacFarland} seguindo o proposto por Rhodes e colaboradores \cite{rhodes2015}.


\subsection{Indução de resistência}

As cepas foram induzidas a resistência com doses sub-letais de antibiótico através de cultivo em meio BHI suplementado com antibióticos,
modificando o protocolo utilizado por Hoeksema e colaboradores com modificação para uso em meio que favoreça o crescimento de \textit{C. Pseudotuberculosis}
\cite{hoeksema2019}.


\subsection{Sequênciamento das amostras}


\subsection{Chamada, predição e avaliação dos variantes}

\section{Estrutura do Trabalho}

% ---
% Capitulo com exemplos de comandos inseridos de arquivo externo 
% ---
\include{abntex2-modelo-include-comandos}
% ---

