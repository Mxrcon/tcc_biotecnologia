\chapter{Introdução}
\label{cap:introducao}

\section{Contexto}
- Necessidade de novos Compostos

- Uso de Biotecnologia para solução de problemas industriais

- Diversidade amazônica como reservatório de descobertas

\section{Justificativa}
Bactérias ambientais são interessantes alvos para a descoberta de compostos
de relevância biotecnológica, especialmente como solução para os crescentes níveis
de resistência a antimicrobianos encontrados em microorganismos patogênicos.
A caracterização genômica e prospecção de genes de interessedesses microorganismos,
especialmente do ambiente amazônico, são um passos importantes
em busca de compostos de potencial farmacológico e industrial.

\chapter{Objetivos}

\section{Objetivo Geral}

predizer o potencial biotecnológico de bactéria,s
provenientes do solo do parque do utinga 

\section{Objetivos Específicos}
\begin{enumerate}
    \item Identificar e caracterizar os organismos em nível de espécie
    \item Determinar o perfil clonal e a distância filogenética entre os isolados
    \item Categorizar os microorganismos quanto a produção de compostos bactericidas.
\end{enumerate}





