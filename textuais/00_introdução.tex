\chapter{Introdução}
\label{cap:introducao}

\section{Contexto}

A infecção de \textit{Coryneacterium pseudotuberculosis} pode causar patologias em equinos, manifestando-se em três formas diferentes: linfangite ulcerativa e abscessos externos e internos \cite{aleman1996}, estas infecções são tratadas com antibióticos de diferentes classes, dentre eles a rifampicina e enroflaxacina. 
Individuos dessa espécie apresentaram altas taxas de sensibilidade \cite{rhodes2015}, porém o uso de antibióticos em rebanhos pode levar a mudanças no atual quadro
da espécie. 

\section{Justificativa}
O presente estudo justifica-se como exploratório pois busca induzir o desenvolviment mecanismos de resitência a antibióticos na espécie 
\textit{Coryneacterium pseudotuberculosis biovar equi} que permancem pouco estudados devido aos altos níveis de sensibilidade
encontrados na espécie, o conhecimento a respeito desses mecanismos é de grande valor para uma melhor compreenção da relação da espécie com fármacos.

\section{Objetivos}

\subsection{Objetivo Geral}

Identificar quais mutações no genoma da C. pseudotuberculosis MB122,MB154 MB302, MB336; estão associadas a resistência aos antibióticos enrofloxacina e rifampicina.

\subsection{Objetivos Específicos}
\begin{enumerate}
    \item Induzir cepas selvagens a resistência aos antibióticos enrofloxacina e rifampicina
    \item Sequenciar o DNA dos isolados multirresistentes
    \item Predizer mutações nos genomas sequenciados utilizando abordagens computacionais
\end{enumerate}

\section{Metodologia}

\subsection{Obtenção das amostras}

Cepas \textit{wild type} de \textit{Coryneacterium pseudotuberculosis biovar equi} foram gentilmente cedidas pelo Prof. Dr. Vasco Azevedo 
da Universidade Federal de Minas Gerais, estas cepas são identificadas pelas sigas MB122, MB154, MB302 e MB336, e seus genomas estão
depositados no banco de dados NCBI sob os códigos de acesso: TODO. As amostras foram isoladas, genotipadas e sequenciadas em um estudo anterior \cite{barauna2017}.
A pureza destas amostras foi verificada através de sequênciamento (TODO MARCA DO SEQUENCIADOR) dos fragmentos resultantes da 
amplificação das regiões V3 e V4 do rna ribissomal 16s, utilizando os primers 8F e 1492R.



\subsection{Determinação da sensibilidade a antibióticos}

A sensibilidade aos antibióticos enroflaxacina e rifampicina foram determinadas utilizando teste de concentração inibitória mínima em microdilução em 
caldo mueller hinton de acordo com o descrito nas
normas M100 e MD45 do CLSI (\citeyear{clsi2015},\citeyear{clsi2020}). Utilizando pré inóculo em meio \textit{Brain Heart Infusion} adicionado de \textit{tween} 80 0.2\%
e ajustado para 0.5 na escala \textit{MacFarland} seguindo o proposto por Rhodes e colaboradores (\citeyear{rhodes2015}).
As cepas foram submetidas a dois testes de sensibilidade, inicialmente após a confirmação da pureza das amostras, e após a indução de resistência por cultivo em placa.


\subsection{Indução de resistência}

As cepas foram induzidas a resistência com doses sub-letais de antibiótico através de cultivo em meio BHI sólido suplementado com antibióticos,
modificando o protocolo utilizado por Hoeksema e colaboradores (\citeyear{hoeksema2019}) com modificação para uso em meio que favoreça o crescimento de \textit{C. Pseudotuberculosis}. 
As cepas foram consideradas resistentes quando o valor inibitório mínimo foi superior a 4 vezes o valor inicial.


\subsection{Sequênciamento das amostras}

Após a obtenção de clones resistentes, seus materiais genéticos foram extraídos utilizando o kit (TODO KIT DE EXTRAÇÂO), e sequenciados utilizando a plataforma (TODO PLATAFORMA DE SEQUENCIAMENTO) seguindo os protocolos dos fabricantes.

\subsection{Chamada, predição e avaliação dos variantes}

As leituras advindas do sequênciamento foram avaliadas utilizando os programas FASTQC e MULTIQC e posteriormente filtradas com a ferramenta TRIMMOMATIC, posteriormente
os variantes foram detectados utilizando os genomas das amostras \textit{wild type} como base para a chamada com o software GATK, posteriormente o software SnpEff foi utilizado para predição funcional dos variantes. 

O \textit{score} SHIFT foi utilizado para avaliação quanto a significância das mutações preditas para a estrutura proteica traduzida a partir do rna transcrito a partir das sequências mutantes.


\section{Estrutura do Trabalho}

% ---
% Capitulo com exemplos de comandos inseridos de arquivo externo 
% ---
\include{abntex2-modelo-include-comandos}
% ---

