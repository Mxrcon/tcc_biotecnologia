\chapter{Introdução}
\label{cap:introducao}

\section{Contexto}



\section{Justificativa}


\section{Objetivos}

\subsection{Objetivo Geral}

Identificar a produção de substâncias antimicrobianas em Actinomicetos

\subsection{Objetivos Específicos}
\begin{enumerate}
    \item alguma coisa a ver com identificação das amostras %TODO
    \item alguma coisa a ver com perfil clonal %TODO
    \item alguma coisa a ver com subst. %TODO
\end{enumerate}

\section{Metodologia}
\subsection{Coleta das amostras}
\subsection{REP}
\begin{enumerate}
    \item Primers%TODO
    \item Gel analyzer e Rscript%TODO
    \item Análise do dendograma%TODO
\end{enumerate}

\subsection{Teste de bioatividade}
\subsubsection{Obtenção dos extratos}
Bactérias obtidos em estudo prévio, foram cultivados em meio Ágar TSB para
reativação. Após isso, os isolados foram inoculados em Caldo TSB por 16 horas,
esse caldo foi centrifugado a 9500G por 8 minutos e o sobrenadante foi coletado
e reservado para uso no mesmo dia.
\subsubsection{Teste de difusão em poço}
Foi utilizado teste de difusão em poço para determinação de atividade antimicrobiana,
para isso foi utilizado meio Müeller-Hinton Ágar, com furos de 10mm de diâmetro.
Em cada furo foi depositado 70\mu L de extrato, as cepas utilizadas foram:
%Corynebacterium fimi NTC 5, Bacillus subtilis ATCC 6633 e Salmonella tyohimmerium ATCC 14028.
\textit{Bacillus Substilis ATCC 6633}, \textit{Corynebacterium Fimi NTC5}, \textit{Staphylococcus Aureus ATCC 25922}
e um isolado resistente de \textit{Klebsiella sp.} obtida em estudo anterior.
Para cada teste, foi preparado inóculo em solução salina 1\% com ajuste para 0.5
na escala McFarland, essa solução foi uniformemente espalhada com uso de swab estéril.

Como controles positivos foram utilizados soluções de antibióticos com
eficácia para cada uma das cepas controle, conforme descrito nas normas CLSI\cite{clsi2020}.

Os halos de inibição foram medidos e análisados estatisticamente utilizando teste de \rho com
o uso da ferramenta R.



\section{Estrutura do Trabalho}

% ---
% Capitulo com exemplos de comandos inseridos de arquivo externo 
% ---
\include{abntex2-modelo-include-comandos}
% ---

